% Options for packages loaded elsewhere
\PassOptionsToPackage{unicode}{hyperref}
\PassOptionsToPackage{hyphens}{url}
%
\documentclass[
  10pt,
  a4paper]{article}
\title{Analysis 1A --- Tutorial 5}
\author{Christian Jones: University of Bath}
\date{November 2023}

\usepackage{amsmath,amssymb}
\usepackage{lmodern}
\usepackage{iftex}
\ifPDFTeX
  \usepackage[T1]{fontenc}
  \usepackage[utf8]{inputenc}
  \usepackage{textcomp} % provide euro and other symbols
\else % if luatex or xetex
  \usepackage{unicode-math}
  \defaultfontfeatures{Scale=MatchLowercase}
  \defaultfontfeatures[\rmfamily]{Ligatures=TeX,Scale=1}
\fi
% Use upquote if available, for straight quotes in verbatim environments
\IfFileExists{upquote.sty}{\usepackage{upquote}}{}
\IfFileExists{microtype.sty}{% use microtype if available
  \usepackage[]{microtype}
  \UseMicrotypeSet[protrusion]{basicmath} % disable protrusion for tt fonts
}{}
\makeatletter
\@ifundefined{KOMAClassName}{% if non-KOMA class
  \IfFileExists{parskip.sty}{%
    \usepackage{parskip}
  }{% else
    \setlength{\parindent}{0pt}
    \setlength{\parskip}{6pt plus 2pt minus 1pt}}
}{% if KOMA class
  \KOMAoptions{parskip=half}}
\makeatother
\usepackage{xcolor}
\IfFileExists{xurl.sty}{\usepackage{xurl}}{} % add URL line breaks if available
\IfFileExists{bookmark.sty}{\usepackage{bookmark}}{\usepackage{hyperref}}
\hypersetup{
  pdftitle={Analysis 1A --- Tutorial 5},
  pdfauthor={Christian Jones: University of Bath},
  hidelinks,
  pdfcreator={LaTeX via pandoc}}
\urlstyle{same} % disable monospaced font for URLs
\usepackage[margin=2.5cm]{geometry}
\usepackage{longtable,booktabs,array}
\usepackage{calc} % for calculating minipage widths
% Correct order of tables after \paragraph or \subparagraph
\usepackage{etoolbox}
\makeatletter
\patchcmd\longtable{\par}{\if@noskipsec\mbox{}\fi\par}{}{}
\makeatother
% Allow footnotes in longtable head/foot
\IfFileExists{footnotehyper.sty}{\usepackage{footnotehyper}}{\usepackage{footnote}}
\makesavenoteenv{longtable}
\usepackage{graphicx}
\makeatletter
\def\maxwidth{\ifdim\Gin@nat@width>\linewidth\linewidth\else\Gin@nat@width\fi}
\def\maxheight{\ifdim\Gin@nat@height>\textheight\textheight\else\Gin@nat@height\fi}
\makeatother
% Scale images if necessary, so that they will not overflow the page
% margins by default, and it is still possible to overwrite the defaults
% using explicit options in \includegraphics[width, height, ...]{}
\setkeys{Gin}{width=\maxwidth,height=\maxheight,keepaspectratio}
% Set default figure placement to htbp
\makeatletter
\def\fps@figure{htbp}
\makeatother
\setlength{\emergencystretch}{3em} % prevent overfull lines
\providecommand{\tightlist}{%
  \setlength{\itemsep}{0pt}\setlength{\parskip}{0pt}}
\setcounter{secnumdepth}{5}
\newcommand{\BOO}{BOO}
\usepackage {hyperref}
\hypersetup {colorlinks = true, linkcolor = blue, urlcolor = blue}
\usepackage{float}
\ifLuaTeX
  \usepackage{selnolig}  % disable illegal ligatures
\fi

\usepackage{amsthm}
\theoremstyle{plain}
\newtheorem*{theorem*}{Theorem}\newtheorem{theorem}{Theorem}[section]
\theoremstyle{definition}
\newtheorem*{definition*}{Definition}\newtheorem{definition}{Definition}[section]
\theoremstyle{plain}
\newtheorem*{proposition*}{Proposition}\newtheorem{proposition}[theorem]{Proposition}
\newtheorem*{Definitions*}{Definitions}\newtheorem{Definitions}[definition]{Definitions}
\theoremstyle{plain}
\newtheorem*{lemma*}{Lemma}\newtheorem{lemma}{Lemma}[section]
\theoremstyle{plain}
\newtheorem*{corollary*}{Corollary}\newtheorem{corollary}{Corollary}[section]
\theoremstyle{plain}
\newtheorem*{conjecture*}{Conjecture}\newtheorem{conjecture}{Conjecture}[section]
\theoremstyle{definition}
\newtheorem*{example*}{Example}\newtheorem{example}{Example}[section]
\theoremstyle{definition}
\newtheorem*{exercise*}{Exercise}\newtheorem{exercise}{Exercise}[section]
\newtheorem*{Non-theorem*}{Non-theorem}\newtheorem{Non-theorem}{Non-theorem}[section]
\newtheorem*{Thought*}{Thought}\newtheorem{Thought}{Thought}[section]
\theoremstyle{remark}
\newtheorem*{remark*}{Remark}
\newtheorem*{solution*}{Solution}
\newtheorem*{Example*}{Example}
\theoremstyle{remark}
\newtheorem*{Proof*}{Proof}
\newtheorem*{Examples*}{Examples}
\let\BeginKnitrBlock\begin \let\EndKnitrBlock\end
\begin{document}
\maketitle

{
\setcounter{tocdepth}{2}
\tableofcontents
}
\newpage
\pagenumbering{arabic}

\hypertarget{introduction}{%
\section*{Introduction}\label{introduction}}
\addcontentsline{toc}{section}{Introduction}

Here is the material to accompany the Analysis Tutorial in Week 5. Alternative formats can be downloaded by clicking the download icon at the top of the page. As usual, send comments and corrections to \href{mailto:caj50@bath.ac.uk}{Christian Jones (caj50)}. To return to the homepage, click \href{http://caj50.github.io/tutoring.html}{here}.

\hypertarget{lecture-recap}{%
\section{Lecture Recap}\label{lecture-recap}}

\hypertarget{sequences}{%
\subsection{Sequences}\label{sequences}}

\hypertarget{two-useful-theorems}{%
\subsubsection{Two Useful Theorems}\label{two-useful-theorems}}

Last week, we were introduced to the idea of sequences and what it means for a sequence to \emph{converge}. The definition of this is repeated below.

\BeginKnitrBlock{definition}[Sequence Convergence]
{\label{def:def1} }A sequence \((a_n)_{n\in\mathbb{N}}\) converges to a real number \(L\) as \(n \longrightarrow \infty\), written as either \(a_n \longrightarrow L\), or \(\lim_{n \to \infty}a_n = L\) if \[\forall \epsilon > 0, \; \exists N = N(\epsilon) \in \mathbb{N}, \; \text{such that} \; \forall n \geq N, \; \lvert a_n - L \rvert < \epsilon.\]
\EndKnitrBlock{definition}

Using this definition, we can establish two theorems which will really help us when looking for further results about sequences.

\BeginKnitrBlock{theorem}[Preservation of Non-Strict Inequalities]
{\label{thm:thm1} }Let \((a_n)_{n\in\mathbb{N}}\) and \((b_n)_{n\in\mathbb{N}}\) be sequences and let \(L,M \in \mathbb{R}\) be such that \(a_n \to L\) and \(b_n \to L\) as \(n \to \infty\). If \(a_n \leq b_n \; \forall n \in \mathbb{N}\), then \(L \leq M\).
\EndKnitrBlock{theorem}
There are two good uses for this theorem. The first says that non-negative sequences should have non-negative limits (which is something you might expect). Before we state the second, we mention one more thing, which is \textbf{not true}:
\BeginKnitrBlock{Non-theorem*}
{} Let \((a_n)_{n\in\mathbb{N}}\) and \((b_n)_{n\in\mathbb{N}}\) be sequences and let \(L,M \in \mathbb{R}\) be such that \(a_n \to L\) and \(b_n \to L\) as \(n \to \infty\). If \(a_n < b_n \; \forall n \in \mathbb{N}\), then \(L < M\).
\EndKnitrBlock{Non-theorem*}
To see why this is false, consider the sequences defined by \(a_n = 1 - \frac{1}{n}\) and \(b_n = 1\). We note that each \(a_n\) is strictly less than each corresponding \(b_n\), but we find that \[\lim_{n \to \infty} a_n = 1 = \lim_{n \to \infty} b_n.\]

The second reason why Theorem \ref{thm:thm1} is so important, is that it gives us this second theorem\footnote{Feel free to ignore this footnote, but there are areas of maths where limits are not unique. This is usually in the realm of topology, which you can take in Year 3 \href{https://www.bath.ac.uk/catalogues/2023-2024/ma/MA30055.html}{(MA30055)}. Luckily for us, everything behaves nicely, and our limits are unique.}:
\BeginKnitrBlock{theorem}[Uniqueness of Limits]
{\label{thm:thm2} }If \((a_n)_{n\in\mathbb{N}}\) is convergent with \(a_n \to L\) and \(a_n \to M\) as \(n \to \infty\), then \(L = M\).
\EndKnitrBlock{theorem}

\hypertarget{bounded-sequences}{%
\subsubsection{Bounded Sequences}\label{bounded-sequences}}

Much like we have done with sets, we can formulate a definition which allows us to `trap' sequences.

\BeginKnitrBlock{definition}[Bounded Sequence]
{\label{def:def2} }A sequence \((a_n)\) is bounded if there exists \(M \in \mathbb{R}\) such that \(\lvert a_n \rvert \leq M\).
\EndKnitrBlock{definition}
If you prefer to think diagramatically, this says we can trap the sequence within a strip of width \(2M\) centred round \(0\). More importantly, this leads to the idea that \emph{all convergent sequences are bounded}. Note that this is equivalent to saying that if a sequence is not bounded, then it is not convergent.

\hypertarget{algebra-of-limits}{%
\subsubsection{Algebra of Limits}\label{algebra-of-limits}}

Using the definition to prove all limits would be an incredibly boring way to go through this course. Luckily, there are a few general results we can prove which make our lives so much easier. This is known as the \emph{algebra of limits} (AoL).

\BeginKnitrBlock{theorem}[Algebra of Limits]
{\label{thm:thm4} }Let \(A,B,c \in \mathbb{R}\) and let \((a_n)\) and \((b_n)\) be sequences with \(a_n \to A\) and \(b_n \to B\) as \(n \to \infty\). Then:

\begin{enumerate}
\def\labelenumi{\arabic{enumi}.}
\tightlist
\item
  \(\lim_{n \to \infty} (a_n + b_n) = A + B\),
\item
  \(\lim_{n \to \infty} (ca_n) = cA\),
\item
  \(\lim_{n \to \infty} (a_n b_n) = AB\),
\item
  If \(b_n \neq 0 \; \forall n \in \mathbb{N}\) and \(B \neq 0\), \(\lim_{n \to \infty} \frac{a_n}{b_n} = \frac{A}{B}\).
\end{enumerate}
\EndKnitrBlock{theorem}

\hypertarget{hints}{%
\section{Hints}\label{hints}}

As per usual, here's where you'll find the problem sheet hints!

\begin{enumerate}
\def\labelenumi{\arabic{enumi}.}
\item
  \begin{enumerate}
  \def\labelenumii{\roman{enumii})}
  \tightlist
  \item
    At this stage of the course, you'll have to use the definition of convergence. Apply this, and follow the hint on the sheet.
  \item
    This one you can apply AoL! However, you should explain why you can use AoL, namely: which convergence results from lectures are you applying?
  \end{enumerate}
\item
  Again, this is one you have to use the definition on (for the time being, at least). Remember, if you make the denominator of a fraction smaller, you'll make the overall fraction bigger.
\item
  For the first bit, you can use AoL! For the second, you'll have to use the definition, making a \textbf{specific} choice of \(\epsilon.\)
\item
  This is very similar to the first question from tutorials, so here's a few (vague-ish) hints:

  \begin{enumerate}
  \def\labelenumii{\roman{enumii})}
  \tightlist
  \item
    Try and find expressions to simplify the sums.
  \item
    Again, you'll want to find an explicit expression for the sum of square numbers.
  \item
    Firstly, what do you get if you factorise \(a^3 - b^3\)? The result of Tutorial Question 2 will also be useful here.
  \item
    Evaluate \(\cos(2n\pi).\)
  \item
    Simplify the expression first.
  \item
    You'll have to use the definition again here.
  \end{enumerate}
\item
  This is another definition question. Use the knowledge that \((a_n)\) converges (i.e.~what does this mean explicitly?) to conclude something about the convergence of \((b_n)\).
\end{enumerate}

\end{document}
