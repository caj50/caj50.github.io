% Options for packages loaded elsewhere
\PassOptionsToPackage{unicode}{hyperref}
\PassOptionsToPackage{hyphens}{url}
%
\documentclass[
  10pt,
  a4paper]{article}
\title{Analysis 1A --- Supplementary Paper 2020}
\author{Christian Jones: University of Bath}
\date{January 2023}

\usepackage{amsmath,amssymb}
\usepackage{lmodern}
\usepackage{iftex}
\ifPDFTeX
  \usepackage[T1]{fontenc}
  \usepackage[utf8]{inputenc}
  \usepackage{textcomp} % provide euro and other symbols
\else % if luatex or xetex
  \usepackage{unicode-math}
  \defaultfontfeatures{Scale=MatchLowercase}
  \defaultfontfeatures[\rmfamily]{Ligatures=TeX,Scale=1}
\fi
% Use upquote if available, for straight quotes in verbatim environments
\IfFileExists{upquote.sty}{\usepackage{upquote}}{}
\IfFileExists{microtype.sty}{% use microtype if available
  \usepackage[]{microtype}
  \UseMicrotypeSet[protrusion]{basicmath} % disable protrusion for tt fonts
}{}
\makeatletter
\@ifundefined{KOMAClassName}{% if non-KOMA class
  \IfFileExists{parskip.sty}{%
    \usepackage{parskip}
  }{% else
    \setlength{\parindent}{0pt}
    \setlength{\parskip}{6pt plus 2pt minus 1pt}}
}{% if KOMA class
  \KOMAoptions{parskip=half}}
\makeatother
\usepackage{xcolor}
\IfFileExists{xurl.sty}{\usepackage{xurl}}{} % add URL line breaks if available
\IfFileExists{bookmark.sty}{\usepackage{bookmark}}{\usepackage{hyperref}}
\hypersetup{
  pdftitle={Analysis 1A --- Supplementary Paper 2020},
  pdfauthor={Christian Jones: University of Bath},
  hidelinks,
  pdfcreator={LaTeX via pandoc}}
\urlstyle{same} % disable monospaced font for URLs
\usepackage[margin=2.5cm]{geometry}
\usepackage{longtable,booktabs,array}
\usepackage{calc} % for calculating minipage widths
% Correct order of tables after \paragraph or \subparagraph
\usepackage{etoolbox}
\makeatletter
\patchcmd\longtable{\par}{\if@noskipsec\mbox{}\fi\par}{}{}
\makeatother
% Allow footnotes in longtable head/foot
\IfFileExists{footnotehyper.sty}{\usepackage{footnotehyper}}{\usepackage{footnote}}
\makesavenoteenv{longtable}
\usepackage{graphicx}
\makeatletter
\def\maxwidth{\ifdim\Gin@nat@width>\linewidth\linewidth\else\Gin@nat@width\fi}
\def\maxheight{\ifdim\Gin@nat@height>\textheight\textheight\else\Gin@nat@height\fi}
\makeatother
% Scale images if necessary, so that they will not overflow the page
% margins by default, and it is still possible to overwrite the defaults
% using explicit options in \includegraphics[width, height, ...]{}
\setkeys{Gin}{width=\maxwidth,height=\maxheight,keepaspectratio}
% Set default figure placement to htbp
\makeatletter
\def\fps@figure{htbp}
\makeatother
\setlength{\emergencystretch}{3em} % prevent overfull lines
\providecommand{\tightlist}{%
  \setlength{\itemsep}{0pt}\setlength{\parskip}{0pt}}
\setcounter{secnumdepth}{5}
\newcommand{\BOO}{BOO}
\usepackage {hyperref}
\hypersetup {colorlinks = true, linkcolor = blue, urlcolor = blue}
\usepackage{float}
\ifLuaTeX
  \usepackage{selnolig}  % disable illegal ligatures
\fi

\usepackage{amsthm}
\theoremstyle{plain}
\newtheorem*{theorem*}{Theorem}\newtheorem{theorem}{Theorem}[section]
\theoremstyle{definition}
\newtheorem*{definition*}{Definition}\newtheorem{definition}{Definition}[section]
\theoremstyle{plain}
\newtheorem*{proposition*}{Proposition}\newtheorem{proposition}[theorem]{Proposition}
\newtheorem*{Definitions*}{Definitions}\newtheorem{Definitions}[definition]{Definitions}
\theoremstyle{plain}
\newtheorem*{lemma*}{Lemma}\newtheorem{lemma}{Lemma}[section]
\theoremstyle{plain}
\newtheorem*{corollary*}{Corollary}\newtheorem{corollary}{Corollary}[section]
\theoremstyle{plain}
\newtheorem*{conjecture*}{Conjecture}\newtheorem{conjecture}{Conjecture}[section]
\theoremstyle{definition}
\newtheorem*{example*}{Example}\newtheorem{example}{Example}[section]
\theoremstyle{definition}
\newtheorem*{exercise*}{Exercise}\newtheorem{exercise}{Exercise}[section]
\newtheorem*{Non-theorem*}{Non-theorem}\newtheorem{Non-theorem}{Non-theorem}[section]
\newtheorem*{Question*}{Question}\newtheorem{Question}{Question}[section]
\newtheorem*{Thought*}{Thought}\newtheorem{Thought}{Thought}[section]
\theoremstyle{remark}
\newtheorem*{remark*}{Remark}
\newtheorem*{solution*}{Solution}
\newtheorem*{Example*}{Example}
\theoremstyle{remark}
\newtheorem*{Proof*}{Proof}
\newtheorem*{Examples*}{Examples}
\let\BeginKnitrBlock\begin \let\EndKnitrBlock\end
\begin{document}
\maketitle

{
\setcounter{tocdepth}{2}
\tableofcontents
}
\newpage
\pagenumbering{arabic}

\hypertarget{introduction}{%
\section*{Introduction}\label{introduction}}
\addcontentsline{toc}{section}{Introduction}

Here are the solutions to the past paper discussed in the revision session on 9th January 2023. This is designed as a guide to how much to write in the exam, and how you might want to style your solutions. To return to the homepage, click \href{http://caj50.github.io/tutoring.html}{here}.

\hypertarget{question-1}{%
\section*{Question 1}\label{question-1}}
\addcontentsline{toc}{section}{Question 1}

\BeginKnitrBlock{Question*}
{}For each of the following concepts, give an example that satisfies the definition and an example that does not. (You need not give any proofs.)

\begin{enumerate}
\def\labelenumi{\alph{enumi})}
\tightlist
\item
  A Cauchy sequence.
\item
  A decreasing sequence.
\item
  A sequentially continuous function.
\item
  A conditionally convergent series.
\item
  An interval.
\end{enumerate}
\EndKnitrBlock{Question*}

\BeginKnitrBlock{solution*}
\begin{enumerate}
\def\labelenumi{\alph{enumi})}
\tightlist
\item
  An example is the sequence \((a_n)_{n\in\mathbb{N}}\), where \(a_n = \frac{1}{n}\). A non-example is the sequence \((b_n)_{n\in\mathbb{N}}\) where \(b_n = n\).
\item
  An example is the sequence \((a_n)_{n\in\mathbb{N}}\), where \(a_n = \frac{1}{n}\). A non-example is the sequence \((b_n)_{n\in\mathbb{N}}\) where \(b_n = n\).
\item
  An example is the function \(f:[0,1] \to \mathbb{R}\) defined by \(f(x) = x\). A non-example is the function \(g:[0,1] \to \mathbb{R}\), where \[g(x) = \begin{cases}
  0 \;\; \text{if}\;\; 0\leq x < 0.5,\\
  1 \;\; \text{if}\;\; 0.5\leq x \leq 1.
  \end{cases}\]
\item
  An example is the series \(\sum_{n=1}^{\infty}\frac{(-1)^n}{n}.\) A non-example is \(\sum_{n=1}^{\infty}\frac{(-1)^n}{n^2}\).
\item
  An example of an interval is the set \(S_1 = (0,1)\). A non-example is the set \(S_2 = (-1,0)\cup(1,2).\)
\end{enumerate}

(If question 1 is like this in the exam, examples which can be used in more than one part will help you save time!)
\EndKnitrBlock{solution*}

\hypertarget{question-2}{%
\section*{Question 2}\label{question-2}}
\addcontentsline{toc}{section}{Question 2}

\BeginKnitrBlock{Question*}
{}The following statements paraphrase theorems, corollaries, propositions, or lemmas from the lectures. Identify them by their names.

\begin{enumerate}
\def\labelenumi{\alph{enumi})}
\tightlist
\item
  Let \((a_n)_{n\in\mathbb{N}}\) be a real sequence. If \(\sup_{n\in\mathbb{N}}\lvert a_n \rvert < \infty\), then there exists a sequence \((n_k)_{k\in\mathbb{N}}\) such that \[\forall k\in\mathbb{N}:\; (n_k \in \mathbb{N})\wedge(n_{k+1} > n_k)\] and there exists \(B \in \mathbb{R}\) such that \[\forall \epsilon > 0 \;\exists K \in \mathbb{N}\;\forall k \geq K: \; \lvert a_{n_{k}} - B \rvert < \epsilon.\]
\item
  Suppose that \((a_n)_{n\in\mathbb{N}}\) and \((s_n)_{n\in\mathbb{N}}\) are two real sequences such that \[s_n = \sum_{n=1}^{\infty} (-1)^ka_k\] for all \(n\in\mathbb{N}\). If \(a_{n+1} \leq a_n\) for all \(n \in \mathbb{N}\) and \(a_n \to 0\) as \(n \to \infty\), then \((s_n)_{n \in \mathbb{N}}\) converges.
\item
  \(\forall x \in \mathbb{R}\;\exists k \in \mathbb{N}: \; k > x.\)
\item
  Let \((x_n)_{n\in\mathbb{N}}\) and \((y_n)_{n\in\mathbb{N}}\) be two sequences such that \[\forall n \in \mathbb{N}: x_n \leq x_{n+1} \leq y_{n+1} \leq y_n.\] Then, there exists \(a \in \mathbb{R}\) such that \[\forall n \in \mathbb{N}: \; x_n \leq a \leq y_n.\]
\item
  Suppose that \(a \in [-1\,\infty)\) and \(k \in \mathbb{N}_0\). Then \[1 + ka \leq (1+a)^k.\]
\end{enumerate}
\EndKnitrBlock{Question*}

\BeginKnitrBlock{solution*}
\begin{enumerate}
\def\labelenumi{\alph{enumi})}
\tightlist
\item
  This is the \emph{Bolzano-Weierstrass} theorem.
\item
  This is the \emph{Leibniz alternating series test} for series.
\item
  This is the \emph{Archimedian Postulate}.
\item
  This is the \emph{Nested Intervals Theorem}.
\item
  This is the \emph{Binomial inequality}.
\end{enumerate}
\EndKnitrBlock{solution*}

\hypertarget{question-3}{%
\section*{Question 3}\label{question-3}}
\addcontentsline{toc}{section}{Question 3}

\BeginKnitrBlock{Question*}
{}Let \((a_n)_{n\in\mathbb{N}}\) be a real sequence and \(L \in \mathbb{R}\).

\begin{enumerate}
\def\labelenumi{\alph{enumi})}
\item
  Show that \(a_n \to \infty\) if and only if \(-a_n \to -\infty\) as \(n \to \infty\).
\item
  Assuming that \(\lim_{n\to\infty}a_n = L\), show that \(\left(\lvert a_n \rvert\right)_{n\in\mathbb{N}}\) does \emph{not} diverge to \(\infty.\)
\item
  \begin{enumerate}
  \def\labelenumii{\roman{enumii})}
  \tightlist
  \item
    Use the growth factor test to show that \[\lim_{n\to\infty}\frac{n^n}{(n!)^2} = 0.\] You may use without proof that \(\lim_{n\to\infty}\left(1 + \frac{1}{n}\right)^n\) exists.
  \item
    Show that there exists \(N \in \mathbb{N}\) such that \[n! \leq n^n \leq \left(\frac{n!}{100}\right)^2,\] for all \(n \in \mathbb{N}\).
  \end{enumerate}
\end{enumerate}

In the following questions (d) and (e), you may use any result from the lectures without proof.

\begin{enumerate}
\def\labelenumi{\alph{enumi})}
\setcounter{enumi}{3}
\tightlist
\item
  Find \[\lim_{n\to\infty}\sqrt{n}(\sqrt{n+1} - \sqrt{n}).\]
\item
  Show that \[\lim_{n\to\infty} \sqrt[n]{2} = 1.\]
\end{enumerate}
\EndKnitrBlock{Question*}

\BeginKnitrBlock{solution*}
\begin{enumerate}
\def\labelenumi{\alph{enumi})}
\item
  We have that \begin{align*} a_n \to \infty &\Longleftrightarrow \forall M \in \mathbb{R}, \; \exists N \in \mathbb{N} \; \text{such that} \;\, a_n \geq M,\\
  &\Longleftrightarrow \forall M \in \mathbb{R}, \; \exists N \in \mathbb{N} \; \text{such that} \;\, -a_n \leq -M.\end{align*} Setting \(K = -M\) in this last statement gives \begin{align*} a_n \to \infty &\Longleftrightarrow \forall K \in \mathbb{R}, \; \exists N \in \mathbb{N} \; \text{such that} \;\, -a_n \leq K, \\
  &\Longleftrightarrow -a_n \to -\infty,\end{align*} as required.
\item
  We claim that \(\lim_{n\to\infty}\lvert a_n \rvert = \lvert L \rvert.\) To this end, fix \(\epsilon >0\). Since \(\lim_{n \to \infty} a_n = L\), we know that there exists \(N \in \mathbb{N}\) such that \[\lvert a_n - L \rvert < \epsilon \;\; \forall n \geq N.\] Now, for all \(n \geq N,\) \begin{align*} \left\lvert \lvert a_n \rvert - \lvert L \rvert \right\rvert &\leq \lvert a_n - L \rvert, \; \; \text{(by the reverse triangle inequality)}\\
  &< \epsilon \end{align*} Hence, since \(\epsilon\) was arbitrary, we conclude that \(\lim_{n\to\infty}\lvert a_n \rvert = \lvert L \rvert.\) In particular, \(\left(\lvert a_n \rvert\right)_{n\in\mathbb{N}}\) does not diverge to \(\infty.\)
\item
  \begin{enumerate}
  \def\labelenumii{\roman{enumii})}
  \tightlist
  \item
    Setting \(b_n = \frac{n^n}{(n!)^2}\) for \(n \in \mathbb{N}\), we see that \(b_n \geq 0\), and \begin{align*}\frac{b_{n+1}}{b_n} &= \frac{(n+1)^{n+1}}{((n+1)!)^2}\frac{(n!)^2}{n^n}\\
    &= (n+1)\frac{(n+1)^n}{n^n}\left(\frac{n!}{(n+1)!}\right)^2\\
    &= (n+1)\left(1 + \frac{1}{n}\right)^n\frac{1}{(n+1)^2}\\
    &= \frac{1/n}{1 + 1/n}\left(1 + \frac{1}{n}\right)^n.\end{align*} Since \(\frac{1}{n} \to 0\), and \(\left(1 + \frac{1}{n}\right)^n \to \mathrm{e}\) as \(n \to \infty\), we find by the algebra of limits that \[\frac{b_{n+1}}{b_n} \to 0\cdot\mathrm{e} = 0 \;\; \text{as}\;\; n \to \infty.\] Since \(0 < 1\), we find by the growth factor test that \[\lim_{n\to\infty}\frac{n^n}{(n!)^2} = 0,\] as required.
  \item
    By part i) and the definition of convergence, we know that \(\exists N \in \mathbb{N}\) such that \(\forall n \geq N\), \[\left\lvert\frac{n^n}{(n!)^2} - 0 \right\rvert =  \frac{n^n}{(n!)^2} \leq \left(\frac{1}{100}\right)^2.\] Also, note that since \(n! \leq n^n\) for all \(n \in \mathbb{N},\) \[\frac{1}{n!} \leq \frac{n^n}{(n!)^2}.\] Hence, for all \(n \geq N,\) \[\frac{1}{n!} \leq \frac{n^n}{(n!)^2} \leq \left(\frac{1}{100}\right)^2 \Longleftrightarrow n! \leq n^n \leq \left(\frac{n!}{100}\right)^2,\] as required.
  \end{enumerate}
\item
  First, note that via completing the square, \begin{align}\sqrt{n}(\sqrt{n+1} - \sqrt{n}) &= \frac{\sqrt{n}(n+1 - n)}{\sqrt{n+1} + \sqrt{n}}\nonumber\\ &= \frac{\sqrt{n}}{\sqrt{n+1} + \sqrt{n}}\nonumber\\ &= \frac{1}{\sqrt{1 + \frac{1}{n}} + 1}.\tag{*}\end{align} Now, we claim that \[\lim_{n \to \infty} \sqrt{1 + \frac{1}{n}} = 1.\] To show this, we fix \(\epsilon > 0\) and consider for \(n \in \mathbb{N}:\) \[\left\lvert \sqrt{1 + \frac{1}{n}} - 1\right\rvert = \frac{1 + \frac{1}{n}-1}{\sqrt{1 + 1/n}+1} \leq \frac{1}{n}.\] We then have that \[\frac{1}{n} < \epsilon \Longleftrightarrow n > \frac{1}{\epsilon}.\] By the Archimedian Postulate, we know there exists \(N \in \mathbb{N}\) such that \(N > \frac{1}{\epsilon}.\) Hence, for any \(n \geq N,\) \[\left\lvert \sqrt{1 + \frac{1}{n}} - 1\right\rvert \leq \frac{1}{n} \leq \frac{1}{N} < \epsilon.\] Therefore, as \(\epsilon > 0\) was arbitrary, \[\lim_{n \to \infty} \sqrt{1 + \frac{1}{n}} = 1.\] Returning to (*) and applying the algebra of limits, we find that as \(n \to \infty\), \[\lim_{n \to \infty}\sqrt{n}(\sqrt{n+1} - \sqrt{n}) = \frac{1}{1+1} = \frac{1}{2}.\]
\item
  Since \(2 > 1\), we write \(\sqrt[n]{2} = 1 + x_n,\) where \(x_n \geq 0\). This gives \begin{align*}
  2 = (1 + x_n)^n &\geq 1 + nx_n \;\; \text{(by the binomial inequality)}\\ &\geq 1. \end{align*} Rearranging, we find \[0 \leq x_n \leq \frac{1}{n}.\] Now, since \(0 \to 0\) and \(\frac{1}{n} \to 0\) as \(n \to \infty\), \(x_n \to 0\) by the sandwich theorem. Hence, by the algebra of limits, \[\lim_{n \to \infty} \sqrt[n]{2} = \lim_{n\to\infty}(1 + x_n) = 1 + 0 = 1,\] as required.
\end{enumerate}
\EndKnitrBlock{solution*}

\hypertarget{question-4}{%
\section*{Question 4}\label{question-4}}
\addcontentsline{toc}{section}{Question 4}

\BeginKnitrBlock{Question*}
{}In this question, you may use any result from the lectures without proof.

\begin{enumerate}
\def\labelenumi{\alph{enumi})}
\tightlist
\item
  Let \(a \in (0,1)\). Using the theorem on the Cauchy product of series, or otherwise, show that \[\left(\sum_{n=0}^{\infty} a^n\right)^2 = \sum_{n=0}^{\infty} (n+1)a^n.\]
\item
  Find the radii of convergence of the following power series.

  \begin{enumerate}
  \def\labelenumii{\roman{enumii})}
  \tightlist
  \item
    \[\sum_{n=0}^{\infty}x^n.\]
  \item
    \[\sum_{n=1}^{\infty}\frac{x^n}{n^n}.\]
  \item
    \[\sum_{n=0}^{\infty}\left(\frac{2\pi}{13}\right)^n\cos\left(\frac{2\pi n}{13}\right)x^n.\]
  \end{enumerate}
\item
  Say whether or not the following series converge and explain your reasoning.

  \begin{enumerate}
  \def\labelenumii{\roman{enumii})}
  \tightlist
  \item
    \[\sum_{n=1}^{\infty}\frac{1}{\sqrt{n}}.\]
  \item
    \[\sum_{n=1}^{\infty}\frac{n}{2^n + n^2}.\]
  \item
    \[\sum_{n=1}^{\infty}\frac{1}{n\log(n)}.\]
  \end{enumerate}
\end{enumerate}
\EndKnitrBlock{Question*}

\BeginKnitrBlock{solution*}
\begin{enumerate}
\def\labelenumi{\alph{enumi})}
\item
  Recall that for \(a \in (0,1)\), the sum is a geometric series with \[\sum_{n=0}^{\infty} a^n = \frac{1}{1-a}.\] As \(\lvert a^n \rvert = a^n \; \forall n \in \mathbb{N}_0\), this series is absolutely convergent, so the Cauchy multiplication theorem gives that \[\left(\sum_{n=0}^{\infty} a^n\right)^2 = \sum_{n=0}^{\infty}c_n,\] where for \(n \in \mathbb{N}_0,\) \[c_n = \sum_{k=0}^{n} a^k a^{n-k} = \sum_{k=0}^{n} a^n = (n+1)a^n.\] Hence \[\left(\sum_{n=0}^{\infty} a^n\right)^2 = \sum_{n=0}^{\infty} (n+1)a^n,\] as required.
\item
  \begin{enumerate}
  \def\labelenumii{\roman{enumii})}
  \tightlist
  \item
    Writing \[\sum_{n=0}^{\infty} x^n = \sum_{n=0}^{\infty} a_n x^n,\] with \(a_n = 1\), we calculate the radius of convergence, \(R\), as \[R = \lim_{n \to \infty} \frac{\lvert a_{n}\rvert}{\lvert a_{n+1}\rvert} = \lim_{n\to\infty} \frac{1}{1} = 1.\]
  \item
    Writing \[\sum_{n=1}^{\infty} \frac{x^n}{n^n} = \sum_{n=1}^{\infty} b_n x^n,\] with \(b_n = \frac{1}{n^n}\), we calculate \[\limsup_{n \to \infty}\lvert b_n \rvert^{1/n} = \limsup_{n\to \infty}\frac{1}{n} = 0.\] Hence, by Cauchy-Hadamard, the radius of convergence \(R\) for this power series is \(R = \infty\).
  \item
    Writing \[\sum_{n=0}^{\infty}\left(\frac{2\pi}{13}\right)^n\cos\left(\frac{2\pi n}{13}\right)x^n = \sum_{n=0}^{\infty} c_n x^n,\] with \(c_n = \left(\frac{2\pi}{13}\right)^n\cos\left(\frac{2\pi n}{13}\right)\), we calculate \[\limsup_{n \to \infty}\left\lvert c_n\right\rvert^{1/n} = \limsup_{n \to \infty}\frac{2\pi}{13}\left\lvert\cos\left(\frac{2\pi n}{13}\right)\right\rvert^{1/n}.\] Setting \(d_n = \frac{2\pi}{13}\left\lvert\cos\left(\frac{2\pi n}{13}\right)\right\rvert^{1/n},\) we claim that \(\limsup_{n \to \infty} d_n = \frac{2\pi}{13}.\) First, as for all \(y \in \mathbb{R}\), \(\lvert \cos(y)\rvert \leq 1\), we see that \(d_n \leq \frac{2\pi}{13}\). This means that \[\limsup_{n\to\infty}d_n \leq \frac{2\pi}{13}.\] Moreover, taking the subsequence \((d_{n_k})_{k\in\mathbb{N}}\), where \(n_k = 13k\), we see that as \(k \to \infty,\) \[d_{13k} = \frac{2\pi}{13}\left\lvert\cos\left(2\pi k \right)\right\rvert^{\frac{1}{13k}} = \frac{2\pi}{13}\cdot 1 \to \frac{2\pi}{13} \;\; \text{(by AoL.)}\] So, \(\limsup_{n \to \infty} d_n \geq \frac{2\pi}{13}\), from which we conclude that \(\limsup_{n \to \infty} d_n = \frac{2\pi}{13}.\) Hence, by Cauchy-Hadamard, the radius of convergence \(R\) for this power series is given by \[R = \left(\limsup_{n \to \infty} d_n\right)^{-1} = \frac{13}{2\pi}.\]
  \end{enumerate}
\item
  \begin{enumerate}
  \def\labelenumii{\roman{enumii})}
  \tightlist
  \item
    Recall from lectures that the series \[\sum_{n=1}^{\infty} \frac{1}{n^{\alpha}}\;\;\begin{cases}\text{diverges if}\;\; \alpha \leq 1,\\
    \text{converges if}\;\; \alpha > 1.
    \end{cases}\] Since \(\sqrt{n} = n^{1/2}\), and \(\frac{1}{2}<1\), we know that the series \(\sum_{n=1}^{\infty}\frac{1}{\sqrt{n}}\) diverges.
  \item
    First, note that
    \begin{align}
    \frac{n}{2^n + n^2} \leq \frac{n}{2^n}.\tag{**}
    \end{align}
    Now, setting \(x_n = \frac{n}{2^n}\), we find \[\frac{\lvert x_{n+1}\rvert}{\lvert x_n \rvert} = \frac{(n+1)2^n}{2^{n+1}n} = \frac{1}{2}\left(1 + \frac{1}{n}\right).\] Hence, by the algebra of limits, \[\frac{\lvert x_{n+1}\rvert}{\lvert x_n \rvert} \to \frac{1}{2} < 1,\] as \(n \to \infty\). So, by d'Alembert's ratio test, we find that \[\sum_{n=1}^{\infty}\frac{n}{2^n} \; \; \text{converges.}\] Finally, by the comparison test as applied to (**), we conclude that \[\sum_{n=1}^{\infty}\frac{n}{2^n + n^2} \; \; \text{converges.}\]
  \item
    Setting \(y_n = \frac{1}{n\log(n)}\), we define for \(k \geq 1,\) \[z_{k}:= 2^k y_{2^k} = \frac{2^k}{2^k\log(2^k)} = \frac{1}{k\log(2)}.\] Using the result stated in part i), we know that as \(\sum_{k=1}^{\infty}\frac{1}{k}\) diverges, \(\sum_{k=1}^{\infty} z_k\) diverges. Hence, by the Cauchy condensation test, the given series \(\sum_{n=1}^{\infty}y_n\) diverges.
  \end{enumerate}
\end{enumerate}
\EndKnitrBlock{solution*}

\end{document}
