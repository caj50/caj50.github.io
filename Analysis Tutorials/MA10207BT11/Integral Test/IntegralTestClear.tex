% Options for packages loaded elsewhere
\PassOptionsToPackage{unicode}{hyperref}
\PassOptionsToPackage{hyphens}{url}
%
\documentclass[
  12pt,
  a4paper]{extarticle}
\title{Analysis 1B --- Integral Test}
\author{Christian Jones: University of Bath}
\date{May 2023}

\usepackage{amsmath,amssymb}
\usepackage{lmodern}
\usepackage{iftex}
\ifPDFTeX
  \usepackage[T1]{fontenc}
  \usepackage[utf8]{inputenc}
  \usepackage{textcomp} % provide euro and other symbols
\else % if luatex or xetex
  \usepackage{unicode-math}
  \defaultfontfeatures{Scale=MatchLowercase}
  \defaultfontfeatures[\rmfamily]{Ligatures=TeX,Scale=1}
\fi
% Use upquote if available, for straight quotes in verbatim environments
\IfFileExists{upquote.sty}{\usepackage{upquote}}{}
\IfFileExists{microtype.sty}{% use microtype if available
  \usepackage[]{microtype}
  \UseMicrotypeSet[protrusion]{basicmath} % disable protrusion for tt fonts
}{}
\makeatletter
\@ifundefined{KOMAClassName}{% if non-KOMA class
  \IfFileExists{parskip.sty}{%
    \usepackage{parskip}
  }{% else
    \setlength{\parindent}{0pt}
    \setlength{\parskip}{6pt plus 2pt minus 1pt}}
}{% if KOMA class
  \KOMAoptions{parskip=half}}
\makeatother
\usepackage{xcolor}
\IfFileExists{xurl.sty}{\usepackage{xurl}}{} % add URL line breaks if available
\IfFileExists{bookmark.sty}{\usepackage{bookmark}}{\usepackage{hyperref}}
\hypersetup{
  pdftitle={Analysis 1B --- Integral Test},
  pdfauthor={Christian Jones: University of Bath},
  hidelinks,
  pdfcreator={LaTeX via pandoc}}
\urlstyle{same} % disable monospaced font for URLs
\usepackage[margin=2.5cm]{geometry}
\usepackage{longtable,booktabs,array}
\usepackage{calc} % for calculating minipage widths
% Correct order of tables after \paragraph or \subparagraph
\usepackage{etoolbox}
\makeatletter
\patchcmd\longtable{\par}{\if@noskipsec\mbox{}\fi\par}{}{}
\makeatother
% Allow footnotes in longtable head/foot
\IfFileExists{footnotehyper.sty}{\usepackage{footnotehyper}}{\usepackage{footnote}}
\makesavenoteenv{longtable}
\usepackage{graphicx}
\makeatletter
\def\maxwidth{\ifdim\Gin@nat@width>\linewidth\linewidth\else\Gin@nat@width\fi}
\def\maxheight{\ifdim\Gin@nat@height>\textheight\textheight\else\Gin@nat@height\fi}
\makeatother
% Scale images if necessary, so that they will not overflow the page
% margins by default, and it is still possible to overwrite the defaults
% using explicit options in \includegraphics[width, height, ...]{}
\setkeys{Gin}{width=\maxwidth,height=\maxheight,keepaspectratio}
% Set default figure placement to htbp
\makeatletter
\def\fps@figure{htbp}
\makeatother
\setlength{\emergencystretch}{3em} % prevent overfull lines
\providecommand{\tightlist}{%
  \setlength{\itemsep}{0pt}\setlength{\parskip}{0pt}}
\setcounter{secnumdepth}{5}
\newcommand{\BOO}{BOO}
\usepackage {hyperref}
\hypersetup {colorlinks = true, linkcolor = blue, urlcolor = blue}
\usepackage{float}
\ifLuaTeX
  \usepackage{selnolig}  % disable illegal ligatures
\fi

\usepackage{amsthm}
\theoremstyle{plain}
\newtheorem*{theorem*}{Theorem}\newtheorem{theorem}{Theorem}[section]
\theoremstyle{definition}
\newtheorem*{definition*}{Definition}\newtheorem{definition}{Definition}[section]
\theoremstyle{plain}
\newtheorem*{proposition*}{Proposition}\newtheorem{proposition}[theorem]{Proposition}
\newtheorem*{Definitions*}{Definitions}\newtheorem{Definitions}[definition]{Definitions}
\theoremstyle{plain}
\newtheorem*{lemma*}{Lemma}\newtheorem{lemma}{Lemma}[section]
\theoremstyle{plain}
\newtheorem*{corollary*}{Corollary}\newtheorem{corollary}{Corollary}[section]
\theoremstyle{plain}
\newtheorem*{conjecture*}{Conjecture}\newtheorem{conjecture}{Conjecture}[section]
\theoremstyle{definition}
\newtheorem*{example*}{Example}\newtheorem{example}{Example}[section]
\theoremstyle{definition}
\newtheorem*{exercise*}{Exercise}\newtheorem{exercise}{Exercise}[section]
\newtheorem*{Non-theorem*}{Non-theorem}\newtheorem{Non-theorem}{Non-theorem}[section]
\newtheorem*{Question*}{Question}\newtheorem{Question}{Question}[section]
\newtheorem*{Thought*}{Thought}\newtheorem{Thought}{Thought}[section]
\theoremstyle{remark}
\newtheorem*{remark*}{Remark}
\newtheorem*{solution*}{Solution}
\newtheorem*{Example*}{Example}
\theoremstyle{remark}
\newtheorem*{Proof*}{Proof}
\newtheorem*{Examples*}{Examples}
\let\BeginKnitrBlock\begin \let\EndKnitrBlock\end


%\usepackage[english,shorthands=off]{babel}
\usepackage{etoolbox}
\usepackage{spverbatim}
\makeatletter
\@ifpackageloaded{float}{}{\usepackage{float}}
\@ifpackageloaded{adjustbox}{}{\usepackage[Export]{adjustbox}}
\makeatother
\floatplacement{figure}{H}
\newcommand{\scalefactor}{1.2}
\adjustboxset*{min width=\scalefactor\width,max width=\linewidth}
\renewcommand{\familydefault}{phv}
\fontfamily{phv}\selectfont
\renewcommand{\em}{\bf}\renewcommand{\textit}{\textbf}\renewcommand{\emph}{\textbf}\renewcommand{\it}{\bf}\renewcommand{\itshape}{\bf}
\setlength{\parindent}{0.0pt}
\setlength{\parskip}{1.0\baselineskip}
\renewcommand{\baselinestretch}{1.5}\selectfont
\setlength{\mathsurround}{0.2em}
\setlength{\arraycolsep}{0.5cm}\renewcommand{\arraystretch}{1.5}
\addtolength{\jot}{\baselineskip}
\renewcommand{\;}{\,}
\sloppy
\allowdisplaybreaks
\usepackage{amsthm}
\newtheoremstyle{plain}{20pt}{3pt}{}{}{\bfseries}{.\newline\nobreak}{1.0em\nobreak}{}
\newtheoremstyle{definition}{20pt}{3pt}{}{}{\bfseries}{.\newline\nobreak}{1.0em\nobreak}{}
\newtheoremstyle{remark}{20pt}{3pt}{}{}{\bfseries}{.\newline\nobreak}{1.0em\nobreak}{}
\csundef{Proof}
\csundef{endProof}
\newenvironment{Proof}
  {\noindent{\bf Proof.}\hspace*{1em}}% Begin
  {\qed\par}% End
%% When redefining an environment it is vital that it has 
%% the same number of arguments as the original
\renewenvironment{proof}[1][\proofname]
  {\trivlist\item\relax\noindent{\bf {#1}.}\hspace*{1em}}% Begin
  {\qed\endtrivlist}% End

\begin{document}
\maketitle

{
\setcounter{tocdepth}{2}
\tableofcontents
}
\newpage
\pagenumbering{arabic}

\hypertarget{introduction}{%
\section*{Introduction}\label{introduction}}
\addcontentsline{toc}{section}{Introduction}

We've reached the end of the course! However, despite their prominance in Analysis 1A, we didn't really say much about infinite series. So, to finish off this semester, I wanted to give you a test for series convergence which we can develop using the theory of integration. This is non-examinable, but the method might come in useful for future courses! Furthermore, the examples here may serve as good practice for unseen exam questions.

As usual, alternative formats can be downloaded by clicking the download icon at the top of the page. Please send any comments or corrections to \href{mailto:caj50@bath.ac.uk}{Christian Jones (caj50)}. To return to the homepage, click \href{http://caj50.github.io/tutoring.html}{here}.

\hypertarget{the-test}{%
\section{The Test}\label{the-test}}

\BeginKnitrBlock{theorem}[Integral Test for Series]
{\label{thm:thm1} }Suppose \((a_n)_{n\in\mathbb{N}}\) is a real sequence. Suppose also that a function \(f\) is positive and decreasing on \([1,\infty)\) and that \(f(n) = a_n\) for all \(n \in \mathbb{N}\). Then, the series \(\sum_{n=1}^{\infty} a_n\) converges if and only if the limit \[\int_1^{\infty} f := \lim_{A \to \infty}\int_1^A f\] exists.
\EndKnitrBlock{theorem}

\BeginKnitrBlock{proof}
Note that the existence of \(\lim_{A \to \infty} \int_1^A f\) is equivalent (by linearity of integration) to the convergence of the series \[\sum_{n=1}^{\infty} \int_{n}^{n+1} f = \int_1^2 f + \int_2^3 f + \int_3^4 f + \ldots.\] Now, since \(f\) is decreasing, for each \(n \in \mathbb{N}\), we can use the subdivision \(P_n = \lbrace n, n+1 \rbrace\) of the intervals \([n, n+1]\) to find
\begin{align}
f(n+1) \leq \int_n^{n+1} f \leq f(n) \tag{*}
\end{align}

Applying the comparison test to the left hand side of (*) shows that if \(\sum_{n=1}^{\infty}\int_n^{n+1} f\) exists, then \(\sum_{n=1}^{\infty} a_{n+1}\) (and hence \(\sum_{n=1}^{\infty}a_n\)) also exists. This proves that \[\lim_{A \to \infty} \int_1^A f\;\;\text{exists} \Longrightarrow \sum_{n=1}^{\infty} a_n\;\;\text{converges.}\]

Finally, applying the comparison test to the right hand side of (*) shows that if \(\sum_{n=1}^{\infty} a_{n}\) exists then \(\sum_{n=1}^{\infty}\int_n^{n+1} f\) also exists. This proves the remaining statement, i.e.~\[\sum_{n=1}^{\infty} a_n\;\;\text{converges} \Longrightarrow  \lim_{A \to \infty} \int_1^A f\;\;\text{exists.}\]
\EndKnitrBlock{proof}

Note that we can replace \(1\) with any \(N \in \mathbb{N}\) in this theorem (such as in the lower series/integral limit), and the resulting modified version of the test still works.

\hypertarget{example}{%
\section{Example}\label{example}}

Providing a result without any practical uses is a bit pointless. So here's an example of this theorem in action! The question(s) here are taken from the textbook \emph{`Calculus'} by Michael Spivak.

\BeginKnitrBlock{Question*}
{}a) Show that \(\int_1^{\infty} \mathrm{e}^y/y^y\;dy\) exists, by considering the series \(\sum_{n=1}^{\infty} (\mathrm{e}/n)^n.\)

\begin{enumerate}
\def\labelenumi{\alph{enumi})}
\setcounter{enumi}{1}
\item
  Show that \[\sum_{n=2}^{\infty}\frac{1}{(\ln(n))^{\ln(n)}}\] converges, by using the integral test. \emph{Hint: use an appropriate substitution and part (a)}.
\item
  Show that \[\sum_{n=2}^{\infty}\frac{1}{(\ln(n))^{\ln(\ln(n))}}\] diverges, by using the integral test. \emph{Hint: Use the same substitution as in part (b), and show directly that the resulting integral diverges.}
\end{enumerate}
\EndKnitrBlock{Question*}

\hypertarget{solutions}{%
\subsection{Solutions}\label{solutions}}

\BeginKnitrBlock{solution*}[Part a]
Firstly, setting \(a_n = (\mathrm{e}/n)^n\), we have \[\left\lvert \frac{a_{n+1}}{a_n}\right\rvert = \frac{\mathrm{e}^{n+1}n^n}{(n+1)^{n+1}\mathrm{e}^n} = \mathrm{e}\cdot\frac{1}{n+1}\cdot\left(1 - \frac{1}{n+1}\right).\] Taking \(n \to \infty\), the algebra of limits gives that as \(n\to\infty\) \[\left\lvert \frac{a_{n+1}}{a_n}\right\rvert \to 0,\] so by d'Alembert's ratio test, the series \(\sum_{n=1}^{\infty} (\mathrm{e}/n)^n\) is convergent.

Now, define \(f:[1,\infty) \to \mathbb{R}\) by \(f(y) = \mathrm{e}^y/y^y.\) Note that \(f\) is strictly decreasing on \([1,\infty)\) and for each \(n\in\mathbb{N}\), \(f(n) = a_n = (\mathrm{e}/n)^n\). Hence, by the integral test, the integral \(\int_1^{\infty} \mathrm{e}^y/y^y\;dy\) exists, as required.
\EndKnitrBlock{solution*}

\BeginKnitrBlock{solution*}[Part b]
Consider the function \(f:[2,\infty) \to \mathbb{R}\) given by \[f(x) = \frac{1}{(\ln(x))^{\ln(x)}}.\] Setting \(y = \ln(x)\), we find that \[\int_2^{\infty} f(x) \; dx = \lim_{A \to \infty}\int_{\mathrm{e}^2}^{\mathrm{e}^A} \frac{1}{y^y}\mathrm{e}^y\;dy,\] which exists by part a). Now, for all \(n \geq 2\), we have that \(f(n) = \frac{1}{(\ln(n))^{\ln(n)}}\). Also, by the chain rule, we find that on \((2,\infty)\), \[f'(x) = -\frac{\ln(\ln(x))+1}{x\ln(x)^{\ln(x)}},\] which is always negative, so \(f\) is decreasing on \([2,\infty).\) Hence, by the integral test, we find that the series \[\sum_{n=2}^{\infty}\frac{1}{(\ln(n))^{\ln(n)}}\] converges.
\EndKnitrBlock{solution*}

\BeginKnitrBlock{solution*}[Part c]
Consider the function \(f:[2,\infty) \to \mathbb{R}\) given by \[f(x) = \frac{1}{(\ln(x))^{\ln(\ln(x))}}.\] By differentiating, we can show that \(f\) is strictly decreasing on \([2,\infty)\), so we can apply the integral test to this function.

Now, setting \(y = \ln(x)\) we have that (if it exists), \[\int_{2}^{\infty} f(x)\;dx = \lim_{A \to \infty} \int_{\mathrm{e}^2}^{\mathrm{e}^A}\frac{1}{y^{\ln(y)}}\mathrm{e}^y\;dy.\] By rules of exponentials, we can rewrite the integrand as \[\frac{\mathrm{e}^y}{y^{\ln(y)}} = \mathrm{e}^{y\left(1 - \frac{\ln^2(y)}{y}\right)}.\]

Writing \(y = \mathrm{e}^z\), we know that (by e.g.~the growth factor test) \[\lim_{y \to \infty}\frac{\ln^2(y)}{y} = \lim_{z \to \infty}\frac{z^2}{\mathrm{e}^z} = 0.\] So by the definition of convergence at \(\infty\) (see Problem Sheet 3), we know that \(\exists M \in [\mathrm{e}^2,\infty)\) such that for all \(y > M\), \[\left\lvert \mathrm{e}^{- \frac{\ln^2(y)}{y}} - 1 \right\rvert < \frac{1}{2}.\]

Rearranging and multiplying by \(\mathrm{e}\), we find \(\forall y > M,\) \[\frac{\mathrm{e}}{2} < \mathrm{e}^{1 - \frac{\ln^2(y)}{y}} < \frac{3\mathrm{e}}{2},\] from which raising everything to the power of \(y\) yields \[\left(\frac{\mathrm{e}}{2}\right)^y < \mathrm{e}^{y\left(1 - \frac{\ln^2(y)}{y}\right)} < \left(\frac{3\mathrm{e}}{2}\right)^y.\] Finally, by properties of the integral, we have that \(\forall y > M,\) and large enough \(A\), \[\int_{M}^{\mathrm{e}^A}\frac{\mathrm{e}^y}{y^{\ln(y)}}\;dy > \int_M^{\mathrm{e}^A}\left(\frac{\mathrm{e}}{2}\right)^y\;dy.\] Using the fundamental theorem of calculus, we can evaluate the right hand integral to obtain \[\int_{M}^{\mathrm{e}^A}\frac{\mathrm{e}^y}{y^{\ln(y)}}\;dy > \frac{1}{1 - \ln(2)}\left[\left(\frac{\mathrm{e}}{2}\right)^{\mathrm{e}^A} - \left(\frac{\mathrm{e}}{2}\right)^{M} \right].\] This right hand side of this inequality diverges as \(A \to \infty\), and since \(\int_{\mathrm{e}^2}^M \frac{\mathrm{e}^y}{y^{\ln(y)}}\;dy\) is finite, the original improper integral \(\int_{2}^{\infty} f(x)\;dx\) also diverges. Hence, by the integral test, the series \[\sum_{n=2}^{\infty}\frac{1}{(\ln(n))^{\ln(\ln(n))}}\] diverges.
\EndKnitrBlock{solution*}

\end{document}
