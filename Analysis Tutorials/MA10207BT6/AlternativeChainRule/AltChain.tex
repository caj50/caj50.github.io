% Options for packages loaded elsewhere
\PassOptionsToPackage{unicode}{hyperref}
\PassOptionsToPackage{hyphens}{url}
%
\documentclass[
  10pt,
  a4paper]{article}
\title{Alternative Chain Rule Proof}
\author{Christian Jones: University of Bath}
\date{March 2023}

\usepackage{amsmath,amssymb}
\usepackage{lmodern}
\usepackage{iftex}
\ifPDFTeX
  \usepackage[T1]{fontenc}
  \usepackage[utf8]{inputenc}
  \usepackage{textcomp} % provide euro and other symbols
\else % if luatex or xetex
  \usepackage{unicode-math}
  \defaultfontfeatures{Scale=MatchLowercase}
  \defaultfontfeatures[\rmfamily]{Ligatures=TeX,Scale=1}
\fi
% Use upquote if available, for straight quotes in verbatim environments
\IfFileExists{upquote.sty}{\usepackage{upquote}}{}
\IfFileExists{microtype.sty}{% use microtype if available
  \usepackage[]{microtype}
  \UseMicrotypeSet[protrusion]{basicmath} % disable protrusion for tt fonts
}{}
\makeatletter
\@ifundefined{KOMAClassName}{% if non-KOMA class
  \IfFileExists{parskip.sty}{%
    \usepackage{parskip}
  }{% else
    \setlength{\parindent}{0pt}
    \setlength{\parskip}{6pt plus 2pt minus 1pt}}
}{% if KOMA class
  \KOMAoptions{parskip=half}}
\makeatother
\usepackage{xcolor}
\IfFileExists{xurl.sty}{\usepackage{xurl}}{} % add URL line breaks if available
\IfFileExists{bookmark.sty}{\usepackage{bookmark}}{\usepackage{hyperref}}
\hypersetup{
  pdftitle={Alternative Chain Rule Proof},
  pdfauthor={Christian Jones: University of Bath},
  hidelinks,
  pdfcreator={LaTeX via pandoc}}
\urlstyle{same} % disable monospaced font for URLs
\usepackage[margin=2.5cm]{geometry}
\usepackage{longtable,booktabs,array}
\usepackage{calc} % for calculating minipage widths
% Correct order of tables after \paragraph or \subparagraph
\usepackage{etoolbox}
\makeatletter
\patchcmd\longtable{\par}{\if@noskipsec\mbox{}\fi\par}{}{}
\makeatother
% Allow footnotes in longtable head/foot
\IfFileExists{footnotehyper.sty}{\usepackage{footnotehyper}}{\usepackage{footnote}}
\makesavenoteenv{longtable}
\usepackage{graphicx}
\makeatletter
\def\maxwidth{\ifdim\Gin@nat@width>\linewidth\linewidth\else\Gin@nat@width\fi}
\def\maxheight{\ifdim\Gin@nat@height>\textheight\textheight\else\Gin@nat@height\fi}
\makeatother
% Scale images if necessary, so that they will not overflow the page
% margins by default, and it is still possible to overwrite the defaults
% using explicit options in \includegraphics[width, height, ...]{}
\setkeys{Gin}{width=\maxwidth,height=\maxheight,keepaspectratio}
% Set default figure placement to htbp
\makeatletter
\def\fps@figure{htbp}
\makeatother
\setlength{\emergencystretch}{3em} % prevent overfull lines
\providecommand{\tightlist}{%
  \setlength{\itemsep}{0pt}\setlength{\parskip}{0pt}}
\setcounter{secnumdepth}{5}
\newcommand{\BOO}{BOO}
\usepackage {hyperref}
\hypersetup {colorlinks = true, linkcolor = blue, urlcolor = blue}
\usepackage{float}
\ifLuaTeX
  \usepackage{selnolig}  % disable illegal ligatures
\fi

\usepackage{amsthm}
\theoremstyle{plain}
\newtheorem*{theorem*}{Theorem}\newtheorem{theorem}{Theorem}[section]
\theoremstyle{definition}
\newtheorem*{definition*}{Definition}\newtheorem{definition}{Definition}[section]
\theoremstyle{plain}
\newtheorem*{proposition*}{Proposition}\newtheorem{proposition}[theorem]{Proposition}
\newtheorem*{Definitions*}{Definitions}\newtheorem{Definitions}[definition]{Definitions}
\theoremstyle{plain}
\newtheorem*{lemma*}{Lemma}\newtheorem{lemma}{Lemma}[section]
\theoremstyle{plain}
\newtheorem*{corollary*}{Corollary}\newtheorem{corollary}{Corollary}[section]
\theoremstyle{plain}
\newtheorem*{conjecture*}{Conjecture}\newtheorem{conjecture}{Conjecture}[section]
\theoremstyle{definition}
\newtheorem*{example*}{Example}\newtheorem{example}{Example}[section]
\theoremstyle{definition}
\newtheorem*{exercise*}{Exercise}\newtheorem{exercise}{Exercise}[section]
\newtheorem*{Thought*}{Thought}\newtheorem{Thought}{Thought}[section]
\theoremstyle{remark}
\newtheorem*{remark*}{Remark}
\newtheorem*{solution*}{Solution}
\newtheorem*{Example*}{Example}
\theoremstyle{remark}
\newtheorem*{Proof*}{Proof}
\newtheorem*{Examples*}{Examples}
\let\BeginKnitrBlock\begin \let\EndKnitrBlock\end
\begin{document}
\maketitle

{
\setcounter{tocdepth}{2}
\tableofcontents
}
\newpage
\pagenumbering{arabic}

\hypertarget{introduction}{%
\section*{Introduction}\label{introduction}}
\addcontentsline{toc}{section}{Introduction}

Here is an alternative proof of the chain rule which uses the \(\epsilon\)-\(\delta\) definition of the limit. It's quite involved, but its a great example if you want more practice with these types of arguments. The typed version here is based off of one presented by Adrian Hill, who previously lectured this course.

\hypertarget{the-chain-rule}{%
\section*{The Chain Rule}\label{the-chain-rule}}
\addcontentsline{toc}{section}{The Chain Rule}

\BeginKnitrBlock{theorem}[Chain Rule]
{\label{thm:thm1} }Let \(g:(a,b) \to \mathbb{R}\) and \(f:(A,B) \to \mathbb{R}\) be such that \(g\left((a,b)\right) \subseteq (A,B).\) Assume that \(g\) is differentiable at \(c\) and \(f\) is differentiable at \(g(c)\). Then the composition \(f\circ g\) is differentiable at \(c\) with \[\left(f\circ g\right)'(c) = f'\left(g(c)\right)g'(c).\]
\EndKnitrBlock{theorem}

\BeginKnitrBlock{proof}
Firstly, note that using the definition of limit, we can recast this problem into the following form: given \(\epsilon > 0\), we seek \(\delta > 0\) such that
\begin{align}
\lvert x - c \rvert < \delta \Rightarrow \left\lvert (f\circ g)(x) - \left[(f\circ g)(c) + f'(g(c))g'(c)(x-c)\right]\right\rvert\leq \epsilon\lvert x - c \rvert \tag{*}
\end{align}

Now, fix \(\epsilon > 0\), and let \(\eta_1, \eta_2 > 0\) be determined later. As \(f\) is differentiable at \(g(c)\), \(\exists \theta_1>0\) such that
\begin{align}
\lvert y - g(c) \rvert < \theta_1 \Rightarrow \left\lvert f(y) - \left[f(g(c)) + f'(g(c))(y - g(c))\right]\right\rvert \leq \eta_1\lvert y - g(c)\rvert \tag{1}
\end{align}

Also, as \(g\) is differentiable at \(c\), \(\exists \theta_2 > 0\) such that
\begin{align}
\lvert x - c \rvert < \theta_2 &\Rightarrow \left\lvert g(x) - \left[g(c) + g'(c)(x - c)\right]\right\rvert \leq \eta_2\lvert x - c\rvert, \tag{2}\\
&\Rightarrow \lvert g(x) - g(c) \rvert \leq \left(\eta_2 + \lvert g'(c) \rvert\right)\lvert x - c \rvert. \tag{3}
\end{align}

So, if \(\lvert x - c \rvert < \delta\) for
\begin{align}
\delta = \min\left\lbrace \theta_2, \frac{\theta_1}{\eta_2 + \lvert g'(c)\rvert}\right\rbrace, \tag{4}
\end{align}
then using (3), we find that

\begin{align*}
\lvert x - c \rvert < \delta \Rightarrow \lvert g(x) - g(c) \rvert < \left(\eta_2 + \lvert g'(c) \rvert\right)\delta \leq \theta_1.
\end{align*}

Substituting \(y = g(x)\) into (1) then gives for \(\lvert x - c \rvert < \delta\),
\begin{align*}
\left\lvert f(g(x)) - \left[f(g(c)) + f'(g(c))(g(x) - g(c))\right]\right\rvert \leq \eta_1\lvert g(x) - g(c) \rvert
\end{align*}

Adding and subtracting \(f'(g(c))g'(c)(x-c)\) yields
\begin{align*}
\left\lvert f(g(x)) - \left[f(g(c))+f'(g(c))g'(c)(x-c)\right] - f'(g(c))\left[g(x) - g(c) - g'(c)(x-c)\right]\right\rvert \leq \eta_1\lvert g(x) - g(c) \rvert.
\end{align*}

Applying the reverse triangle inequality and rearranging,
\begin{align}
\left\lvert f(g(x)) - \left[f(g(c))+f'(g(c))g'(c)(x-c)\right]\right\rvert &\leq \left\lvert f'(g(c)) \right\rvert \lvert g(x) - g(c) - g'(c)(x-c)\rvert \nonumber \\
&\;\;+ \eta_1\lvert g(x) - g(c)\rvert, \nonumber \\
&\leq \eta_2\left\lvert f'(g(c))\right\rvert\lvert x - c \rvert + \eta_1(\eta_2 + \lvert g'(c)\rvert)\lvert x - c \rvert.\nonumber
\end{align}

So, if \[\eta_2 = \frac{\epsilon}{2\left(\lvert f'(g(c))\rvert + 1\right)}\;\;\;\text{and}\;\;\eta_1 = \frac{\epsilon}{2\left(\eta_2 + \lvert g'(c) \rvert\right)},\] then this final inequality implies that for \(\lvert x - c \rvert < \delta\),
\begin{align*}
\left\lvert (f\circ g)(x) - \left[(f\circ g)(c) + f'(g(c))g'(c)(x-c)\right]\right\rvert\leq \left(\frac{\epsilon}{2} + \frac{\epsilon}{2}\right)\lvert x - c \rvert = \epsilon\lvert x - c \rvert
\end{align*}

Hence, provided \(\theta_1\) and \(\theta_2\) are respectively defined for \(\eta_1\) and \(\eta_2\) by (1) and (2), and \(\delta\) is defined by (4), we find that (*) is satisfied, and the result follows.
\EndKnitrBlock{proof}

\end{document}
