% Options for packages loaded elsewhere
\PassOptionsToPackage{unicode}{hyperref}
\PassOptionsToPackage{hyphens}{url}
%
\documentclass[
  17pt,
  a4paper]{extarticle}
\title{Further Examples --- Limits Inferior and Superior}
\author{Christian Jones: University of Bath}
\date{November 2022}

\usepackage{amsmath,amssymb}
\usepackage{lmodern}
\usepackage{iftex}
\ifPDFTeX
  \usepackage[T1]{fontenc}
  \usepackage[utf8]{inputenc}
  \usepackage{textcomp} % provide euro and other symbols
\else % if luatex or xetex
  \usepackage{unicode-math}
  \defaultfontfeatures{Scale=MatchLowercase}
  \defaultfontfeatures[\rmfamily]{Ligatures=TeX,Scale=1}
\fi
% Use upquote if available, for straight quotes in verbatim environments
\IfFileExists{upquote.sty}{\usepackage{upquote}}{}
\IfFileExists{microtype.sty}{% use microtype if available
  \usepackage[]{microtype}
  \UseMicrotypeSet[protrusion]{basicmath} % disable protrusion for tt fonts
}{}
\makeatletter
\@ifundefined{KOMAClassName}{% if non-KOMA class
  \IfFileExists{parskip.sty}{%
    \usepackage{parskip}
  }{% else
    \setlength{\parindent}{0pt}
    \setlength{\parskip}{6pt plus 2pt minus 1pt}}
}{% if KOMA class
  \KOMAoptions{parskip=half}}
\makeatother
\usepackage{xcolor}
\IfFileExists{xurl.sty}{\usepackage{xurl}}{} % add URL line breaks if available
\IfFileExists{bookmark.sty}{\usepackage{bookmark}}{\usepackage{hyperref}}
\hypersetup{
  pdftitle={Further Examples --- Limits Inferior and Superior},
  pdfauthor={Christian Jones: University of Bath},
  hidelinks,
  pdfcreator={LaTeX via pandoc}}
\urlstyle{same} % disable monospaced font for URLs
\usepackage[margin=2.5cm]{geometry}
\usepackage{longtable,booktabs,array}
\usepackage{calc} % for calculating minipage widths
% Correct order of tables after \paragraph or \subparagraph
\usepackage{etoolbox}
\makeatletter
\patchcmd\longtable{\par}{\if@noskipsec\mbox{}\fi\par}{}{}
\makeatother
% Allow footnotes in longtable head/foot
\IfFileExists{footnotehyper.sty}{\usepackage{footnotehyper}}{\usepackage{footnote}}
\makesavenoteenv{longtable}
\usepackage{graphicx}
\makeatletter
\def\maxwidth{\ifdim\Gin@nat@width>\linewidth\linewidth\else\Gin@nat@width\fi}
\def\maxheight{\ifdim\Gin@nat@height>\textheight\textheight\else\Gin@nat@height\fi}
\makeatother
% Scale images if necessary, so that they will not overflow the page
% margins by default, and it is still possible to overwrite the defaults
% using explicit options in \includegraphics[width, height, ...]{}
\setkeys{Gin}{width=\maxwidth,height=\maxheight,keepaspectratio}
% Set default figure placement to htbp
\makeatletter
\def\fps@figure{htbp}
\makeatother
\setlength{\emergencystretch}{3em} % prevent overfull lines
\providecommand{\tightlist}{%
  \setlength{\itemsep}{0pt}\setlength{\parskip}{0pt}}
\setcounter{secnumdepth}{5}
\newcommand{\BOO}{BOO}
\usepackage{float}
\ifLuaTeX
  \usepackage{selnolig}  % disable illegal ligatures
\fi

\usepackage{amsthm}
\theoremstyle{plain}
\newtheorem*{theorem*}{Theorem}\newtheorem{theorem}{Theorem}[section]
\theoremstyle{plain}
\newtheorem*{lemma*}{Lemma}\newtheorem{lemma}{Lemma}[section]
\theoremstyle{plain}
\newtheorem*{corollary*}{Corollary}\newtheorem{corollary}{Corollary}[section]
\theoremstyle{plain}
\newtheorem*{proposition*}{Proposition}\newtheorem{proposition}{Proposition}[section]
\theoremstyle{plain}
\newtheorem*{conjecture*}{Conjecture}\newtheorem{conjecture}{Conjecture}[section]
\theoremstyle{definition}
\newtheorem*{definition*}{Definition}\newtheorem{definition}{Definition}[section]
\theoremstyle{definition}
\newtheorem*{example*}{Example}\newtheorem{example}{Example}[section]
\theoremstyle{definition}
\newtheorem*{exercise*}{Exercise}\newtheorem{exercise}{Exercise}[section]
\theoremstyle{remark}
\newtheorem*{remark*}{Remark}
\newtheorem*{solution*}{Solution}
\let\BeginKnitrBlock\begin \let\EndKnitrBlock\end


%\usepackage[english,shorthands=off]{babel}
\usepackage{etoolbox}
\usepackage{spverbatim}
\makeatletter
\@ifpackageloaded{float}{}{\usepackage{float}}
\@ifpackageloaded{adjustbox}{}{\usepackage[Export]{adjustbox}}
\makeatother
\floatplacement{figure}{H}
\newcommand{\scalefactor}{1.7}
\adjustboxset*{min width=\scalefactor\width,max width=\linewidth}
\renewcommand{\familydefault}{phv}
\fontfamily{phv}\selectfont
\renewcommand{\em}{\bf}\renewcommand{\textit}{\textbf}\renewcommand{\emph}{\textbf}\renewcommand{\it}{\bf}\renewcommand{\itshape}{\bf}
\setlength{\parindent}{0.0pt}
\setlength{\parskip}{1.0\baselineskip}
\renewcommand{\baselinestretch}{1.5}\selectfont
\setlength{\mathsurround}{0.2em}
\setlength{\arraycolsep}{0.5cm}\renewcommand{\arraystretch}{1.5}
\addtolength{\jot}{\baselineskip}
\renewcommand{\;}{\,}
\sloppy
\allowdisplaybreaks
\usepackage{amsthm}
\newtheoremstyle{plain}{20pt}{3pt}{}{}{\bfseries}{.\newline\nobreak}{1.0em\nobreak}{}
\newtheoremstyle{definition}{20pt}{3pt}{}{}{\bfseries}{.\newline\nobreak}{1.0em\nobreak}{}
\newtheoremstyle{remark}{20pt}{3pt}{}{}{\bfseries}{.\newline\nobreak}{1.0em\nobreak}{}
\csundef{Proof}
\csundef{endProof}
\newenvironment{Proof}
  {\noindent{\bf Proof.}\hspace*{1em}}% Begin
  {\qed\par}% End
%% When redefining an environment it is vital that it has 
%% the same number of arguments as the original
\renewenvironment{proof}[1][\proofname]
  {\trivlist\item\relax\noindent{\bf {#1}.}\hspace*{1em}}% Begin
  {\qed\endtrivlist}% End

\begin{document}
\maketitle

{
\setcounter{tocdepth}{2}
\tableofcontents
}
\newpage
\pagenumbering{arabic}

\hypertarget{overview}{%
\section*{Overview}\label{overview}}
\addcontentsline{toc}{section}{Overview}

In this document, you'll find three different examples of finding the limit superior (\(\limsup\)) and limit inferior (\(\liminf\)) of a sequence, with different methods used in each case.

\hypertarget{example-1}{%
\subsection{Example 1}\label{example-1}}

\BeginKnitrBlock{example}
{\label{exm:ex1} }Consider the sequence \((a_n)_{n}\) defined by \[a_n = (-1)^n\frac{2n}{1+3n}.\] Find \(\limsup_{n \to \infty} a_n\) and \(\liminf_{n \to \infty} a_n\).
\EndKnitrBlock{example}

\textbf{Solution}
Firstly, note that we can rewrite each \(a_n\) as \[a_n = (-1)^n \frac{2}{3}\frac{1}{\frac{1}{3n} + 1}.\] Splitting into odd and even cases, we obtain \[a_n = \begin{cases} \frac{2}{3}\frac{1}{\frac{1}{3n} + 1} &\quad \text{for $n$ even},\\
\frac{2}{3}\frac{-1}{\frac{1}{3n} + 1} &\quad \text{for $n$ odd}.\end{cases}\] Note that for \(j \in \mathbb{N}\), \(a_{2j-1} \leq 0 \leq a_{2j}\). Also note that \((a_{2j-1})_j\) is a decreasing sequence and \((a_{2j})_{j}\) is an increasing sequence {[}Try showing these!{]} Moreover, \(\lvert a_n \rvert \leq \frac{2}{3} \; \forall n\in\mathbb{N}\), so \((a_n)_n\) is bounded.

Now, fix \(k \in \mathbb{N}\). We have:
\begin{align*}
\sup_{k\geq n}a_n &= \sup_{2j \geq k} a_{2j}, \; \; \text{(since only 'even' elements are non-negative.)}\\
&= \lim_{j \to \infty} a_{2j}, \; \; \text{(since $(a_{2j})_j$ is a bounded increasing sequence)},\\
&= \frac{2}{3} \; \; \text{(by AoL)}
\end{align*}

\hypertarget{series}{%
\subsection{Series}\label{series}}

It might look like we're done with sequences, but in the grand scheme of things, we're only really getting started. Since with each sequence \((a_n)_{n\in\mathbb{N}}\), we have an infinite list of real numbers, we might consider trying to manipulate them in some way. One way we can do this is by adding them together, which leads to the notion of a \emph{series}.

\BeginKnitrBlock{definition}[Series]
{\label{def:def1} }Let \((a_n)_{n \in \mathbb{N}}\) be a real sequence. Then \[\sum_{n = 1}^{\infty} a_n\] is called a series for \((a_n)_{n\in\mathbb{N}}\).
\EndKnitrBlock{definition}

Much like with sequences, we have an analogous version of convergence for a series:
\BeginKnitrBlock{definition}[Series Convergence and Partial Sums]
{\label{def:def2} }Let \((a_n)_{n \in \mathbb{N}}\) be a real sequence. Then \(\sum_{n = 1}^{\infty} a_n\) converges if and only if the sequence \((S_N)_{N \in \mathbb{N}}\) converges, where \[S_N:= \sum_{n = 1}^{N} a_n\] is the \(N\)\textsuperscript{th} partial sum. If \(S_N \to \ell\) as \(N \to \infty\), we define \[\ell = \sum_{n = 1}^{\infty}a_n.\]
\EndKnitrBlock{definition}
If \((S_N)_{N\in\mathbb{N}}\) diverges to \(\pm\infty\), we say that the corresponding series \[\sum_{n=1}^{\infty} a_n = \pm\infty.\] Finally, if \((S_N)_{N\in\mathbb{N}}\) doesn't converge to a limit, we say that the series diverges without limit.

\hypertarget{algebra-of-series}{%
\subsubsection{Algebra of Series}\label{algebra-of-series}}

By applying the algebra of limits to the sequences of partial sums, we can deduce some handy results.

\BeginKnitrBlock{theorem}[Algebra of Series]
{\label{thm:thm3} }Let \(\sum_{n=1}^{\infty} a_n\) and \(\sum_{n=1}^{\infty} b_n\) be convergent series, and let \(\alpha,\beta \in \mathbb{R}\). Then \[\sum_{n = 1}^{\infty} (\alpha a_n + \beta b_n) = \alpha\sum_{n=1}^{\infty} a_n + \beta\sum_{n=1}^{\infty} b_n.\]
\EndKnitrBlock{theorem}

\hypertarget{some-other-useful-results}{%
\subsubsection{Some Other Useful Results}\label{some-other-useful-results}}

Firstly, we can relate the size of the terms of a series to the overall sum.

\BeginKnitrBlock{proposition}
{\label{prp:prop1} }Let \(\sum_{n=1}^{\infty} a_n\) and \(\sum_{n=1}^{\infty} b_n\) be real series. If \(a_n \leq b_n \, \forall n\in\mathbb{N}\), then \[\sum_{n=1}^{\infty} a_n \leq \sum_{n=1}^{\infty} b_n.\]
\EndKnitrBlock{proposition}

Secondly, we have a \emph{necessary} condition for convergence of a series.

\BeginKnitrBlock{proposition}
{\label{prp:prop2} }Let \(\sum_{n=1}^{\infty} a_n\) be a convergent series. Then \(a_n \to 0\) as \(n \to \infty\).
\EndKnitrBlock{proposition}
Note that the converse of this theorem \emph{does not} hold (think of the sum \(\sum_{n=1}^{\infty} \frac{1}{n}\)). However, the contrapositive is very good at showing that a series does not converge!

\BeginKnitrBlock{proposition}
{\label{prp:prop3} }Let \(\sum_{n=1}^{\infty} a_n\) be a series. If \(a_n \not\to 0\) as \(n \to \infty\), then \(\sum_{n=1}^{\infty} a_n\) does not converge.
\EndKnitrBlock{proposition}

\hypertarget{hints}{%
\section{Hints}\label{hints}}

As per usual, here's where you'll find the problem sheet hints!

\begin{itemize}
\tightlist
\item
  {[}H1.{]} Try using a similar argument to the one used in tutorial question 1 (i.e.~use the fact that the sequence can be split into odd and even cases to your advantage)
\item
  {[}H2.{]} For this question, think about what it means for a series to be convergent. You'll also want to split the terms of the series up in some way. (Think of tutorial question 3a.)
\item
  {[}H3.{]} For the first part, think induction. The only other thing I'll say is to make sure you state all the main results you use!
\end{itemize}

\end{document}
