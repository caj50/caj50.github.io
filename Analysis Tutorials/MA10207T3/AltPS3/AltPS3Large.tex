% Options for packages loaded elsewhere
\PassOptionsToPackage{unicode}{hyperref}
\PassOptionsToPackage{hyphens}{url}
%
\documentclass[
  17pt,
  a4paper]{extarticle}
\title{Analysis 1A --- Tutorial 3}
\author{Christian Jones: University of Bath}
\date{October 2022}

\usepackage{amsmath,amssymb}
\usepackage{lmodern}
\usepackage{iftex}
\ifPDFTeX
  \usepackage[T1]{fontenc}
  \usepackage[utf8]{inputenc}
  \usepackage{textcomp} % provide euro and other symbols
\else % if luatex or xetex
  \usepackage{unicode-math}
  \defaultfontfeatures{Scale=MatchLowercase}
  \defaultfontfeatures[\rmfamily]{Ligatures=TeX,Scale=1}
\fi
% Use upquote if available, for straight quotes in verbatim environments
\IfFileExists{upquote.sty}{\usepackage{upquote}}{}
\IfFileExists{microtype.sty}{% use microtype if available
  \usepackage[]{microtype}
  \UseMicrotypeSet[protrusion]{basicmath} % disable protrusion for tt fonts
}{}
\makeatletter
\@ifundefined{KOMAClassName}{% if non-KOMA class
  \IfFileExists{parskip.sty}{%
    \usepackage{parskip}
  }{% else
    \setlength{\parindent}{0pt}
    \setlength{\parskip}{6pt plus 2pt minus 1pt}}
}{% if KOMA class
  \KOMAoptions{parskip=half}}
\makeatother
\usepackage{xcolor}
\IfFileExists{xurl.sty}{\usepackage{xurl}}{} % add URL line breaks if available
\IfFileExists{bookmark.sty}{\usepackage{bookmark}}{\usepackage{hyperref}}
\hypersetup{
  pdftitle={Analysis 1A --- Tutorial 3},
  pdfauthor={Christian Jones: University of Bath},
  hidelinks,
  pdfcreator={LaTeX via pandoc}}
\urlstyle{same} % disable monospaced font for URLs
\usepackage[margin=2.5cm]{geometry}
\usepackage{longtable,booktabs,array}
\usepackage{calc} % for calculating minipage widths
% Correct order of tables after \paragraph or \subparagraph
\usepackage{etoolbox}
\makeatletter
\patchcmd\longtable{\par}{\if@noskipsec\mbox{}\fi\par}{}{}
\makeatother
% Allow footnotes in longtable head/foot
\IfFileExists{footnotehyper.sty}{\usepackage{footnotehyper}}{\usepackage{footnote}}
\makesavenoteenv{longtable}
\usepackage{graphicx}
\makeatletter
\def\maxwidth{\ifdim\Gin@nat@width>\linewidth\linewidth\else\Gin@nat@width\fi}
\def\maxheight{\ifdim\Gin@nat@height>\textheight\textheight\else\Gin@nat@height\fi}
\makeatother
% Scale images if necessary, so that they will not overflow the page
% margins by default, and it is still possible to overwrite the defaults
% using explicit options in \includegraphics[width, height, ...]{}
\setkeys{Gin}{width=\maxwidth,height=\maxheight,keepaspectratio}
% Set default figure placement to htbp
\makeatletter
\def\fps@figure{htbp}
\makeatother
\setlength{\emergencystretch}{3em} % prevent overfull lines
\providecommand{\tightlist}{%
  \setlength{\itemsep}{0pt}\setlength{\parskip}{0pt}}
\setcounter{secnumdepth}{5}
\newcommand{\BOO}{BOO}
\usepackage {hyperref}
\hypersetup {colorlinks = true, linkcolor = blue, urlcolor = blue}
\usepackage{float}
\ifLuaTeX
  \usepackage{selnolig}  % disable illegal ligatures
\fi

\usepackage{amsthm}
\theoremstyle{plain}
\newtheorem*{theorem*}{Theorem}\newtheorem{theorem}{Theorem}[section]
\theoremstyle{definition}
\newtheorem*{definition*}{Definition}\newtheorem{definition}{Definition}[section]
\theoremstyle{plain}
\newtheorem*{proposition*}{Proposition}\newtheorem{proposition}[theorem]{Proposition}
\newtheorem*{Definitions*}{Definitions}\newtheorem{Definitions}[definition]{Definitions}
\theoremstyle{plain}
\newtheorem*{lemma*}{Lemma}\newtheorem{lemma}{Lemma}[section]
\theoremstyle{plain}
\newtheorem*{corollary*}{Corollary}\newtheorem{corollary}{Corollary}[section]
\theoremstyle{plain}
\newtheorem*{conjecture*}{Conjecture}\newtheorem{conjecture}{Conjecture}[section]
\theoremstyle{definition}
\newtheorem*{example*}{Example}\newtheorem{example}{Example}[section]
\theoremstyle{definition}
\newtheorem*{exercise*}{Exercise}\newtheorem{exercise}{Exercise}[section]
\newtheorem*{Non-theorem*}{Non-theorem}\newtheorem{Non-theorem}{Non-theorem}[section]
\newtheorem*{Thought*}{Thought}\newtheorem{Thought}{Thought}[section]
\theoremstyle{remark}
\newtheorem*{remark*}{Remark}
\newtheorem*{solution*}{Solution}
\newtheorem*{Example*}{Example}
\theoremstyle{remark}
\newtheorem*{Proof*}{Proof}
\newtheorem*{Examples*}{Examples}
\let\BeginKnitrBlock\begin \let\EndKnitrBlock\end


%\usepackage[english,shorthands=off]{babel}
\usepackage{etoolbox}
\usepackage{spverbatim}
\makeatletter
\@ifpackageloaded{float}{}{\usepackage{float}}
\@ifpackageloaded{adjustbox}{}{\usepackage[Export]{adjustbox}}
\makeatother
\floatplacement{figure}{H}
\newcommand{\scalefactor}{1.7}
\adjustboxset*{min width=\scalefactor\width,max width=\linewidth}
\renewcommand{\familydefault}{phv}
\fontfamily{phv}\selectfont
\renewcommand{\em}{\bf}\renewcommand{\textit}{\textbf}\renewcommand{\emph}{\textbf}\renewcommand{\it}{\bf}\renewcommand{\itshape}{\bf}
\setlength{\parindent}{0.0pt}
\setlength{\parskip}{1.0\baselineskip}
\renewcommand{\baselinestretch}{1.5}\selectfont
\setlength{\mathsurround}{0.2em}
\setlength{\arraycolsep}{0.5cm}\renewcommand{\arraystretch}{1.5}
\addtolength{\jot}{\baselineskip}
\renewcommand{\;}{\,}
\sloppy
\allowdisplaybreaks
\usepackage{amsthm}
\newtheoremstyle{plain}{20pt}{3pt}{}{}{\bfseries}{.\newline\nobreak}{1.0em\nobreak}{}
\newtheoremstyle{definition}{20pt}{3pt}{}{}{\bfseries}{.\newline\nobreak}{1.0em\nobreak}{}
\newtheoremstyle{remark}{20pt}{3pt}{}{}{\bfseries}{.\newline\nobreak}{1.0em\nobreak}{}
\csundef{Proof}
\csundef{endProof}
\newenvironment{Proof}
  {\noindent{\bf Proof.}\hspace*{1em}}% Begin
  {\qed\par}% End
%% When redefining an environment it is vital that it has 
%% the same number of arguments as the original
\renewenvironment{proof}[1][\proofname]
  {\trivlist\item\relax\noindent{\bf {#1}.}\hspace*{1em}}% Begin
  {\qed\endtrivlist}% End

\begin{document}
\maketitle

{
\setcounter{tocdepth}{2}
\tableofcontents
}
\newpage
\pagenumbering{arabic}

\hypertarget{introduction}{%
\section*{Introduction}\label{introduction}}
\addcontentsline{toc}{section}{Introduction}

Here is a version of Tutorial Question 3 off of Problem Sheet 3 with an alternative solution for part c). Parts a) and b) are included for completeness.

\BeginKnitrBlock{example}[PS3 Question 3]
{\label{exm:ex1} }a) Show that \[ 2xy \leq x^2 + y^2, \;\; \forall x,y \in \mathbb{R},\] and that equality holds only if \(x = y\).
b) Show that \[\sqrt{\frac{x}{2}} + \sqrt{\frac{y}{2}} \leq \sqrt{x + y} \leq \sqrt{x} + \sqrt{y}, \;\; \forall x,y > 0.\]
c) Prove that \[\lvert \sqrt{1 + x^2} - \sqrt{1 + y^2} \rvert \leq \lvert x - y \rvert \;\; \forall x,y \in \mathbb{R}.\]
\EndKnitrBlock{example}

\hypertarget{part-a}{%
\subsection*{Part a)}\label{part-a}}
\addcontentsline{toc}{subsection}{Part a)}

\BeginKnitrBlock{solution*}
We have that for any \(x,y\in\mathbb{R},\) \[ 0 \leq (x-y)^2 = x^2 - 2xy + y^2,\] from which rearranging gives \[2xy \leq x^2 + y^2.\] Now, \[2xy = x^2 + y^2\; \Leftrightarrow\; 0 = (x-y)^2 \;\Leftrightarrow \;x-y = 0 \;\Leftrightarrow\; x=y.\] So equality holds \emph{if and only if} \(x = y.\)
\EndKnitrBlock{solution*}

\hypertarget{part-b}{%
\subsection*{Part b)}\label{part-b}}
\addcontentsline{toc}{subsection}{Part b)}

\BeginKnitrBlock{solution*}
Consider the second inequality first. Note that since \(x,y > 0\), \(\sqrt{x}, \sqrt{y} > 0\), and so \[x + y \leq x + 2\sqrt{x}\sqrt{y} + y = (\sqrt{x} + \sqrt{y})^2.\] Square rooting this result gives us that \[\sqrt{x + y} \leq \sqrt{x} + \sqrt{y}.\] Next, we have that
\begin{align*}
\left(\sqrt{\frac{x}{2}} + \sqrt{\frac{y}{2}}\right)^2 &= \frac{x}{2} + 2\sqrt{\frac{x}{2}}\sqrt{\frac{y}{2}} + \frac{y}{2},\\
&\leq \frac{x}{2} + 2\left(\frac{x}{2} + \frac{y}{2}\right) + \frac{y}{2} \;\;\; \text{(by part a)},\\
&= x + y.
\end{align*}
Again, square rooting gives us that \[\sqrt{\frac{x}{2}} + \sqrt{\frac{y}{2}} \leq \sqrt{x + y}.\]
\EndKnitrBlock{solution*}

\hypertarget{part-c}{%
\subsection*{Part c)}\label{part-c}}
\addcontentsline{toc}{subsection}{Part c)}

\BeginKnitrBlock{solution*}
Firstly for \(x = -y\), \[\lvert \sqrt{1 + x^2} - \sqrt{1 + y^2}\rvert = 0 \leq \lvert -2y \rvert = \lvert x - y \rvert.\] For \(x \neq -y\), we have
\begin{align}
\lvert \sqrt{1 + x^2} - \sqrt{1 + y^2}\rvert &= \frac{\lvert 1 + x^2 - (1 + y^2) \rvert}{\sqrt{1 + x^2} + \sqrt{1 + y^2}},\tag{*}\\
&= \frac{\lvert x^2 - y^2 \rvert}{\sqrt{1 + x^2} + \sqrt{1 + y^2}},\nonumber\\
&= \frac{\lvert x + y \rvert \lvert x - y \rvert}{\sqrt{1 + x^2} + \sqrt{1 + y^2}}.\nonumber
\end{align}
Now, \[\lvert x \rvert \leq \sqrt{1 + x^2}, \;\;\text{and}\;\; \lvert y \rvert \leq \sqrt{1 + y^2}.\] (This can be seen by squaring both sides of each inequality)

So,
\begin{align*}
\lvert x + y \rvert &\leq \lvert x \rvert + \lvert y \rvert \;\;\; \text{(by the triangle inequality)}\\
&\leq \sqrt{1 + x^2} + \sqrt{1 + y^2},\\
\Leftrightarrow \frac{1}{\lvert x + y \rvert} &\geq \frac{1}{\sqrt{1 + x^2} + \sqrt{1 + y^2}}.
\end{align*}
Therefore,
\begin{align}
\lvert \sqrt{1 + x^2} - \sqrt{1 + y^2}\rvert &= \frac{\lvert x + y \rvert \lvert x - y \rvert}{\sqrt{1 + x^2} + \sqrt{1 + y^2}},\nonumber\\
&\leq \frac{\lvert x - y \rvert\lvert x + y \rvert}{\lvert x + y \rvert},\tag{**}\\
&= \lvert x - y \rvert,\nonumber
\end{align}
as required!
\EndKnitrBlock{solution*}

You might have a few questions about this:

\textbf{Q1)} Why is 3c) done in this way?

A1) It's an alternative way to the one in the model solutions, but I think it's good because it uses some techniques that are useful for the sequences part of the course (e.g.~the triangle inequality and step (*)).

\textbf{Q2)} Where on Earth did the case \(x = -y\) come from?

A2) If you look at (**), this expression doesn't work if \(x = -y\), so you need to consider this separately. It's not an obvious case until you actually reach (**), but once you realise it, it's an easy thing to add to the start of your solution.

\textbf{Q3)} What about (*)? Where does this come from?

A3) Recall for \(a,b \in \mathbb{R}\), \[(a-b)(a+b) = a^2 - b^2.\] Taking \(a = \sqrt{1 + x^2}, \;\; \text{and} \;\; b = \sqrt{1 + y^2},\) we have that \[\sqrt{1 + x^2} - \sqrt{1 + y^2} = \frac{(1+x^2)-(1+y^2)}{\sqrt{1 + x^2} + \sqrt{1 + y^2}}.\]

\end{document}
