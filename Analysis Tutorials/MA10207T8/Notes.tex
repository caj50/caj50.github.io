% Options for packages loaded elsewhere
\PassOptionsToPackage{unicode}{hyperref}
\PassOptionsToPackage{hyphens}{url}
%
\documentclass[
  10pt,
  a4paper]{article}
\title{Analysis 1A --- Tutorial 8}
\author{Christian Jones: University of Bath}
\date{November 2022}

\usepackage{amsmath,amssymb}
\usepackage{lmodern}
\usepackage{iftex}
\ifPDFTeX
  \usepackage[T1]{fontenc}
  \usepackage[utf8]{inputenc}
  \usepackage{textcomp} % provide euro and other symbols
\else % if luatex or xetex
  \usepackage{unicode-math}
  \defaultfontfeatures{Scale=MatchLowercase}
  \defaultfontfeatures[\rmfamily]{Ligatures=TeX,Scale=1}
\fi
% Use upquote if available, for straight quotes in verbatim environments
\IfFileExists{upquote.sty}{\usepackage{upquote}}{}
\IfFileExists{microtype.sty}{% use microtype if available
  \usepackage[]{microtype}
  \UseMicrotypeSet[protrusion]{basicmath} % disable protrusion for tt fonts
}{}
\makeatletter
\@ifundefined{KOMAClassName}{% if non-KOMA class
  \IfFileExists{parskip.sty}{%
    \usepackage{parskip}
  }{% else
    \setlength{\parindent}{0pt}
    \setlength{\parskip}{6pt plus 2pt minus 1pt}}
}{% if KOMA class
  \KOMAoptions{parskip=half}}
\makeatother
\usepackage{xcolor}
\IfFileExists{xurl.sty}{\usepackage{xurl}}{} % add URL line breaks if available
\IfFileExists{bookmark.sty}{\usepackage{bookmark}}{\usepackage{hyperref}}
\hypersetup{
  pdftitle={Analysis 1A --- Tutorial 8},
  pdfauthor={Christian Jones: University of Bath},
  hidelinks,
  pdfcreator={LaTeX via pandoc}}
\urlstyle{same} % disable monospaced font for URLs
\usepackage[margin=2.5cm]{geometry}
\usepackage{longtable,booktabs,array}
\usepackage{calc} % for calculating minipage widths
% Correct order of tables after \paragraph or \subparagraph
\usepackage{etoolbox}
\makeatletter
\patchcmd\longtable{\par}{\if@noskipsec\mbox{}\fi\par}{}{}
\makeatother
% Allow footnotes in longtable head/foot
\IfFileExists{footnotehyper.sty}{\usepackage{footnotehyper}}{\usepackage{footnote}}
\makesavenoteenv{longtable}
\usepackage{graphicx}
\makeatletter
\def\maxwidth{\ifdim\Gin@nat@width>\linewidth\linewidth\else\Gin@nat@width\fi}
\def\maxheight{\ifdim\Gin@nat@height>\textheight\textheight\else\Gin@nat@height\fi}
\makeatother
% Scale images if necessary, so that they will not overflow the page
% margins by default, and it is still possible to overwrite the defaults
% using explicit options in \includegraphics[width, height, ...]{}
\setkeys{Gin}{width=\maxwidth,height=\maxheight,keepaspectratio}
% Set default figure placement to htbp
\makeatletter
\def\fps@figure{htbp}
\makeatother
\setlength{\emergencystretch}{3em} % prevent overfull lines
\providecommand{\tightlist}{%
  \setlength{\itemsep}{0pt}\setlength{\parskip}{0pt}}
\setcounter{secnumdepth}{5}
\newcommand{\BOO}{BOO}
\usepackage{float}
\ifLuaTeX
  \usepackage{selnolig}  % disable illegal ligatures
\fi

\usepackage{amsthm}
\theoremstyle{plain}
\newtheorem*{theorem*}{Theorem}\newtheorem{theorem}{Theorem}[section]
\theoremstyle{plain}
\newtheorem*{lemma*}{Lemma}\newtheorem{lemma}{Lemma}[section]
\theoremstyle{plain}
\newtheorem*{corollary*}{Corollary}\newtheorem{corollary}{Corollary}[section]
\theoremstyle{plain}
\newtheorem*{proposition*}{Proposition}\newtheorem{proposition}{Proposition}[section]
\theoremstyle{plain}
\newtheorem*{conjecture*}{Conjecture}\newtheorem{conjecture}{Conjecture}[section]
\theoremstyle{definition}
\newtheorem*{definition*}{Definition}\newtheorem{definition}{Definition}[section]
\theoremstyle{definition}
\newtheorem*{example*}{Example}\newtheorem{example}{Example}[section]
\theoremstyle{definition}
\newtheorem*{exercise*}{Exercise}\newtheorem{exercise}{Exercise}[section]
\theoremstyle{remark}
\newtheorem*{remark*}{Remark}
\newtheorem*{solution*}{Solution}
\let\BeginKnitrBlock\begin \let\EndKnitrBlock\end
\begin{document}
\maketitle

{
\setcounter{tocdepth}{2}
\tableofcontents
}
\newpage
\pagenumbering{arabic}

\hypertarget{introduction}{%
\section*{Introduction}\label{introduction}}
\addcontentsline{toc}{section}{Introduction}

Here is the material to accompany the 8th Analysis Tutorial on the 28th November. As usual, send comments and corrections to \href{mailto:caj50@bath.ac.uk}{Christian Jones (caj50)}

\hypertarget{lecture-recap}{%
\section{Lecture Recap}\label{lecture-recap}}

\hypertarget{series-convergence}{%
\subsection{Series Convergence}\label{series-convergence}}

Recall from last week that we can define the convergence of an infinite sum/series as follows:

\BeginKnitrBlock{definition}[Series Convergence and Partial Sums]
{\label{def:def1} }Let \((a_n)_{n \in \mathbb{N}}\) be a real sequence. Then \(\sum_{n = 1}^{\infty} a_n\) converges if and only if the sequence \((S_N)_{N \in \mathbb{N}}\) converges, where \[S_N:= \sum_{n = 1}^{N} a_n\] is the \(N\)\textsuperscript{th} partial sum. If \(S_N \to \ell\) as \(N \to \infty\), we define \[\ell = \sum_{n = 1}^{\infty}a_n.\]
\EndKnitrBlock{definition}
Much like with proving sequence convergence, using the definition each time you want to `evaluate' a series can get tedious really quickly. Therefore, we really want a couple of tests which can prove convergence without too much hassle. Before we discuss these tests though, we need to introduce the ideas of \emph{absolute and conditional convergence}.

\BeginKnitrBlock{definition}[Absolute Convergence]
{\label{def:def2} }A real series \(\sum_{n = 1}^{\infty} a_n\) is absolutely convergent if \(\sum_{n = 1}^{\infty} \lvert a_n \rvert\) converges.
\EndKnitrBlock{definition}
For example, if we consider the series \(\sum_{n = 1}^{\infty} a_n\), where \(a_n\) is given by \[a_n = \frac{(-1)^n}{n^2},\] we find that \[\sum_{n = 1}^{\infty} \lvert a_n \rvert = \sum_{n=1}^{\infty} \frac{1}{n^2},\] which we know converges from lectures\footnote{If you take the \emph{Vector Calculus and PDEs} module next year, you'll show that this sum equals \(\frac{\pi^2}{6}\).}. Hence \(\sum_{n = 1}^{\infty} a_n\) is absolutely convergent. Have we learnt anything about the convergence of \(\sum_{n=1}^{\infty}a_n\) here? Turns out the answer is yes, and this is because of the following result.

\BeginKnitrBlock{proposition}
{\label{prp:prop1} }If a real series \(\sum_{n = 1}^{\infty} a_n\) is absolutely convergent, then it is convergent.
\EndKnitrBlock{proposition}
At this stage, we can introduce the idea of conditional convergence too.

\BeginKnitrBlock{definition}[Conditional Convergence]
{\label{def:def3} }Let \(\sum_{n = 1}^{\infty} a_n\) be a real series. If \(\sum_{n = 1}^{\infty} a_n\) is convergent, but \(\sum_{n = 1}^{\infty} \lvert a_n \rvert\) is not, then \(\sum_{n = 1}^{\infty} a_n\) is said to be conditionally convergent.
\EndKnitrBlock{definition}

\hypertarget{series-rearrangement}{%
\subsubsection{Series Rearrangement}\label{series-rearrangement}}

So, what can we do with absolutely convergent series?
\BeginKnitrBlock{theorem}
{\label{thm:thm1} }Suppose \(\sum_{n = 1}^{\infty} a_n\) is an absolutely convergent series, and that \(\sigma: \mathbb{N} \to \mathbb{N}\) is a bijection. Then \(\sum_{n = 1}^{\infty} a_{\sigma(n)}\) is also an absolutely convergent series, and \[\sum_{n = 1}^{\infty} a_n = \sum_{n = 1}^{\infty} a_{\sigma(n)}.\]
\EndKnitrBlock{theorem}
This theorem tells us that for an absolutely convergent series, we can order the terms any way we like, and still reach the same value for the series. At this point, you might be interested to know what happens if we don't have absolute convergence. Long story short, weird things can happen, as is seen in the following theorem.

\BeginKnitrBlock{theorem}[Riemann Rearrangement Theorem]
{\label{thm:thm2} }Suppose \(\sum_{n = 1}^{\infty} a_n\) is conditionally convergent. Then, for any \(\alpha \in \mathbb{R}\), or \(\alpha = \pm\infty\), there exists a bijection \(\sigma:\mathbb{N} \to \mathbb{N}\) such that \[\sum_{n = 1}^{\infty} a_{\sigma(n)} = \alpha.\]
\EndKnitrBlock{theorem}
So what we see here is that we really need to be careful in which order we sum up the terms of a conditionally convergent series!

\hypertarget{tests-for-convergence}{%
\subsection{Tests for Convergence}\label{tests-for-convergence}}

Now that we have the idea of absolute convergence, we can state some convergence tests applicable to series.

\hypertarget{comparison-test}{%
\subsubsection{Comparison Test}\label{comparison-test}}

The first of these tests involves comparing the sizes of two series, and is aptly known as the comparison test.

\BeginKnitrBlock{theorem}[Comparison Test]
{\label{thm:thm3} }Let \((a_n)_n\) and \((b_n)_n\) be real sequences, and suppose that there exists a \(M \in \mathbb{N}\) such that \(\lvert a_n \rvert \leq b_n \;\forall n \geq M.\)
Then, if \(\sum_{n = 1}^{\infty} b_n\) is convergent, \(\sum_{n = 1}^{\infty} a_n\) is convergent.
\EndKnitrBlock{theorem}
Naturally, using this, we can also build a test for divergence to \(\infty\) out of the comparison test too.

\BeginKnitrBlock{corollary}
{\label{cor:corol1} }Let \((a_n)_n\) and \((b_n)_n\) be real sequences. If there exists a \(M \in \mathbb{N}\) such that \(0 \leq a_n \leq b_n \; \forall n \geq M\), and \(\sum_{n = 1}^{\infty} a_n\) diverges, then \(\sum_{n = 1}^{\infty} b_n\) diverges.
\EndKnitrBlock{corollary}
Here, we require the \(a_n\) values to be non-negative to force any divergence of \(\sum_{n = 1}^{\infty} a_n\) to be to \(\infty\). If we allowed, say, \(a_n = (-1)^nn\), then \(\sum_{n = 1}^{\infty} a_n\) would diverge without limit, making this divergence test useless.

\hypertarget{dalemberts-ratio-test}{%
\subsubsection{D'Alembert's Ratio Test}\label{dalemberts-ratio-test}}

This one is quite similar to the growth factor test for sequences, except that due to the idea of absolute convergence (and Proposition \ref{prp:prop1}), the terms of the series only have to be non-zero:

\BeginKnitrBlock{theorem}[D'Alembert's Ratio Test]
{\label{thm:thm4} }Let \((a_n)_n\) be a real sequence with \(a_n \neq 0 \; \forall n \in \mathbb{N}\). Suppose \[\lim_{n\to\infty}\frac{\lvert a_{n+1}\rvert}{\lvert a_n\rvert} = r.\] Then:

\begin{itemize}
\tightlist
\item
  If \(0 \leq r < 1\), \(\sum_{n = 1}^{\infty} a_n\) converges.
\item
  If \(r > 1\), then \(\sum_{n = 1}^{\infty} a_n\) diverges.
\item
  If \(r = 1\), the test is inconclusive.
\end{itemize}
\EndKnitrBlock{theorem}

To see why the test fails for \(r = 1\), consider the three series: \[\sum_{n = 1}^{\infty} \frac{(-1)^{n+1}}{n^2}, \quad \sum_{n = 1}^{\infty} \frac{(-1)^{n+1}}{n} \;\; \text{and} \;\; \sum_{n = 1}^{\infty} (-1)^{n+1}.\] The first is absolutely convergent, the second is conditionally convergent and the third diverges without any limit at all!

\hypertarget{cauchy-condensation-test}{%
\subsubsection{Cauchy Condensation Test}\label{cauchy-condensation-test}}

The final test we're going to look at here is yet another thing named after Cauchy! This one is very good when the terms of a series involve logarithms, and can also be used to show that \[\sum_{n = 1}^{\infty} \frac{1}{n^{\alpha}} \;\;\text{converges} \Longleftrightarrow \alpha > 1.\]

\BeginKnitrBlock{theorem}[Cauchy]
{\label{thm:thm5} }Assume \((a_n)_n\) satisfies \(a_n \geq 0 \; \forall n \in \mathbb{N}\), and is a decreasing sequence. For \(k \in \mathbb{N}\), define \(b_k := 2^ka_{2^k}\). Then \[\sum_{n = 1}^{\infty} a_n \;\; \text{converges}\; \Longleftrightarrow\; \sum_{k = 1}^{\infty} b_k \;\; \text{converges}.\]
\EndKnitrBlock{theorem}

We conclude here with a \href{https://math.stackexchange.com/questions/2071016/does-sum-infty-3-fracn2lnlnnlnn-converge?rq=1}{link} to an example of the Cauchy condensation test in practice. It's highly unlikely you'll ever get something like this in the exam, but the numbers involved are so ridiculous it's worth including here nonetheless!

\hypertarget{hints}{%
\section{Hints}\label{hints}}

As per usual, here's where you'll find the problem sheet hints!

\begin{itemize}
\tightlist
\item
  {[}H1.{]} Think about all the methods you know for proving whether a series converges. Some of the methods from the tutorial may come in handy\ldots{}
\item
  {[}H2.{]} Pretty much the same as homework question 1. However\ldots{}

  \begin{itemize}
  \tightlist
  \item
    {[}H2b.{]} I've got a few pointers for this one. Make sure you know how the binomial coefficient is defined. Also, try to avoid expanding any unnecessary brackets --- if you're writing \(n^3, n^4\) etc. in your solutions, you're putting in more effort than needed!
  \end{itemize}
\item
  {[}H3.{]} This one is only slightly more involved. Know your definitions, and again, think of possible convergence tests to apply.
\end{itemize}

\end{document}
