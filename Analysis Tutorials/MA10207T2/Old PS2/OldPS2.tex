% Options for packages loaded elsewhere
\PassOptionsToPackage{unicode}{hyperref}
\PassOptionsToPackage{hyphens}{url}
%
\documentclass[
  10pt,
  a4paper]{article}
\title{Analysis 1A --- Supremum Example}
\author{Christian Jones: University of Bath}
\date{October 2022}

\usepackage{amsmath,amssymb}
\usepackage{lmodern}
\usepackage{iftex}
\ifPDFTeX
  \usepackage[T1]{fontenc}
  \usepackage[utf8]{inputenc}
  \usepackage{textcomp} % provide euro and other symbols
\else % if luatex or xetex
  \usepackage{unicode-math}
  \defaultfontfeatures{Scale=MatchLowercase}
  \defaultfontfeatures[\rmfamily]{Ligatures=TeX,Scale=1}
\fi
% Use upquote if available, for straight quotes in verbatim environments
\IfFileExists{upquote.sty}{\usepackage{upquote}}{}
\IfFileExists{microtype.sty}{% use microtype if available
  \usepackage[]{microtype}
  \UseMicrotypeSet[protrusion]{basicmath} % disable protrusion for tt fonts
}{}
\makeatletter
\@ifundefined{KOMAClassName}{% if non-KOMA class
  \IfFileExists{parskip.sty}{%
    \usepackage{parskip}
  }{% else
    \setlength{\parindent}{0pt}
    \setlength{\parskip}{6pt plus 2pt minus 1pt}}
}{% if KOMA class
  \KOMAoptions{parskip=half}}
\makeatother
\usepackage{xcolor}
\IfFileExists{xurl.sty}{\usepackage{xurl}}{} % add URL line breaks if available
\IfFileExists{bookmark.sty}{\usepackage{bookmark}}{\usepackage{hyperref}}
\hypersetup{
  pdftitle={Analysis 1A --- Supremum Example},
  pdfauthor={Christian Jones: University of Bath},
  hidelinks,
  pdfcreator={LaTeX via pandoc}}
\urlstyle{same} % disable monospaced font for URLs
\usepackage[margin=2.5cm]{geometry}
\usepackage{longtable,booktabs,array}
\usepackage{calc} % for calculating minipage widths
% Correct order of tables after \paragraph or \subparagraph
\usepackage{etoolbox}
\makeatletter
\patchcmd\longtable{\par}{\if@noskipsec\mbox{}\fi\par}{}{}
\makeatother
% Allow footnotes in longtable head/foot
\IfFileExists{footnotehyper.sty}{\usepackage{footnotehyper}}{\usepackage{footnote}}
\makesavenoteenv{longtable}
\usepackage{graphicx}
\makeatletter
\def\maxwidth{\ifdim\Gin@nat@width>\linewidth\linewidth\else\Gin@nat@width\fi}
\def\maxheight{\ifdim\Gin@nat@height>\textheight\textheight\else\Gin@nat@height\fi}
\makeatother
% Scale images if necessary, so that they will not overflow the page
% margins by default, and it is still possible to overwrite the defaults
% using explicit options in \includegraphics[width, height, ...]{}
\setkeys{Gin}{width=\maxwidth,height=\maxheight,keepaspectratio}
% Set default figure placement to htbp
\makeatletter
\def\fps@figure{htbp}
\makeatother
\setlength{\emergencystretch}{3em} % prevent overfull lines
\providecommand{\tightlist}{%
  \setlength{\itemsep}{0pt}\setlength{\parskip}{0pt}}
\setcounter{secnumdepth}{5}
\newcommand{\BOO}{BOO}
\usepackage {hyperref}
\hypersetup {colorlinks = true, linkcolor = blue, urlcolor = blue}
\usepackage{float}
\ifLuaTeX
  \usepackage{selnolig}  % disable illegal ligatures
\fi

\usepackage{amsthm}
\theoremstyle{plain}
\newtheorem*{theorem*}{Theorem}\newtheorem{theorem}{Theorem}[section]
\theoremstyle{definition}
\newtheorem*{definition*}{Definition}\newtheorem{definition}{Definition}[section]
\theoremstyle{plain}
\newtheorem*{proposition*}{Proposition}\newtheorem{proposition}[theorem]{Proposition}
\newtheorem*{Definitions*}{Definitions}\newtheorem{Definitions}[definition]{Definitions}
\theoremstyle{plain}
\newtheorem*{lemma*}{Lemma}\newtheorem{lemma}{Lemma}[section]
\theoremstyle{plain}
\newtheorem*{corollary*}{Corollary}\newtheorem{corollary}{Corollary}[section]
\theoremstyle{plain}
\newtheorem*{conjecture*}{Conjecture}\newtheorem{conjecture}{Conjecture}[section]
\theoremstyle{definition}
\newtheorem*{example*}{Example}\newtheorem{example}{Example}[section]
\theoremstyle{definition}
\newtheorem*{exercise*}{Exercise}\newtheorem{exercise}{Exercise}[section]
\newtheorem*{Non-theorem*}{Non-theorem}\newtheorem{Non-theorem}{Non-theorem}[section]
\newtheorem*{Order Axioms*}{Order Axioms}\newtheorem{Order Axioms}{Order Axioms}[section]
\newtheorem*{Thought*}{Thought}\newtheorem{Thought}{Thought}[section]
\theoremstyle{remark}
\newtheorem*{remark*}{Remark}
\newtheorem*{solution*}{Solution}
\newtheorem*{Example*}{Example}
\theoremstyle{remark}
\newtheorem*{Proof*}{Proof}
\newtheorem*{Examples*}{Examples}
\let\BeginKnitrBlock\begin \let\EndKnitrBlock\end
\begin{document}
\maketitle

{
\setcounter{tocdepth}{2}
\tableofcontents
}
\newpage
\pagenumbering{arabic}

\hypertarget{question}{%
\section*{Question}\label{question}}
\addcontentsline{toc}{section}{Question}

Here is an example of finding the supremum of a set taken from an old problem sheet, together with three possible methods to find it.
\BeginKnitrBlock{example}
{\label{exm:unnamed-chunk-2} }Let \[B = \left\lbrace \frac{2n-1}{n+1} \lvert n \in \mathbb{N}\right\rbrace.\] Show that \(B\) is bounded above and find \(\sup(B).\)
\EndKnitrBlock{example}

\hypertarget{method-1-contradiction}{%
\subsection*{Method 1 --- Contradiction}\label{method-1-contradiction}}
\addcontentsline{toc}{subsection}{Method 1 --- Contradiction}

\BeginKnitrBlock{solution*}
First, note for \(n\in\mathbb{N}\): \[\frac{2n-1}{n+1} = \frac{2n+2-3}{n+1} = 2 - \frac{3}{n+1} < 2.\] Hence \(B\) is bounded above by \(2\). Therefore, by the completeness axiom, as \(B \neq \emptyset,\) \(\sup(B)\) exists and \(\sup(B) \leq 2.\)

Next, suppose for contradiction that \(\sup(B) < 2\). Now, for any \(x < 2,\) \[2 - \frac{3}{n+1} > x \Leftrightarrow n+1 > \frac{3}{2-x} \Leftrightarrow n > \frac{3}{2-x} - 1.\] Taking \(x = \sup(B)\) and applying Archimedes' Postulate, \(\exists N \in \mathbb{N}\) such that
\begin{align*}
N &> \frac{3}{2-\sup(B)} - 1,\\
\Leftrightarrow 2 - \frac{3}{N+1} &> \sup(B),
\end{align*}
which is a contradiction as \(2 - \frac{3}{N+1} \in B.\) Hence \(\sup(B) \geq 2\), and by combining our found inequalities, \(\sup(B)=2\).
\EndKnitrBlock{solution*}

\hypertarget{method-2-alternative-characterisation-of-suprema}{%
\subsection*{Method 2 --- Alternative Characterisation of Suprema}\label{method-2-alternative-characterisation-of-suprema}}
\addcontentsline{toc}{subsection}{Method 2 --- Alternative Characterisation of Suprema}

\BeginKnitrBlock{solution*}
First, note for \(n\in\mathbb{N}\): \[\frac{2n-1}{n+1} = \frac{2n+2-3}{n+1} = 2 - \frac{3}{n+1} < 2.\] Hence \(B\) is bounded above by \(2\). Therefore as \(B \neq \emptyset,\) the completeness axiom says that \(\sup(B)\) exists.

We claim that \(\sup(B) = 2.\) Fix \(\epsilon > 0.\) Then, for \(n \in \mathbb{N}:\)
\begin{align*}
2 - \frac{3}{n+1} &> 2-\epsilon,\\
\Leftrightarrow \epsilon &> \frac{3}{n+1},\\
\Leftrightarrow n\epsilon &> 3 - \epsilon,\\
\Leftrightarrow n &> \frac{3-\epsilon}{\epsilon}.
\end{align*}
Now, by Archimedes' Postulate, \(\exists N \in \mathbb{N}\) such that \(N > \frac{3-\epsilon}{\epsilon}\), from which \[2 - \frac{3}{N+1} > 2- \epsilon.\] At this stage, take \(b = 2 - \frac{3}{N+1} \in B\). Since \(\epsilon > 0\) was arbitrary, we have that \(\forall \epsilon > 0, \exists b \in B\) such that \(b > 2-\epsilon.\) So, by the alternative characterisation of suprema (Theorem 2.1), \(\sup(B) = 2.\)
\EndKnitrBlock{solution*}

\hypertarget{method-3-limits}{%
\subsection*{Method 3 --- Limits}\label{method-3-limits}}
\addcontentsline{toc}{subsection}{Method 3 --- Limits}

Note that this doesn't work in general, but it might be quicker when you can use it. It relies on the following theorem (which we'll eventually cover):
\BeginKnitrBlock{theorem}
{\label{thm:unnamed-chunk-5} }A bounded, increasing sequence \((b_n)_{n \in \mathbb{N}}\) is convergent, and its limit is given by \[\lim_{n \to \infty} b_n = \sup\lbrace b_n \,\lvert\, n \in \mathbb{N} \rbrace.\]
\EndKnitrBlock{theorem}

\BeginKnitrBlock{solution*}
Define \(b_n = \frac{2n - 1}{n+1}\) for \(n \in \mathbb{N}\).

\textbf{Step 1 --- Show \((b_n)_n\) is bounded above}:

First, note for \(n\in\mathbb{N}\): \[b_n = \frac{2n+2-3}{n+1} = 2 - \frac{3}{n+1} < 2.\] Hence \(B\) is bounded above by \(2\). Therefore, as \(B \neq \emptyset,\) the completeness axiom says that \(\sup(B)\) exists.

\textbf{Step 2 --- Show \((b_n)_n\) is increasing (i.e.~show \(b_{n+1} \geq b_n \; \forall n \in \mathbb{N}\))}:

We have for \(n \in \mathbb{N}\),
\begin{align*}
b_{n+1} - b_{n} &= 2 - \frac{3}{n+2} - \left(2 - \frac{3}{n+1}\right),\\
&= \frac{3(n+2)-3(n+1)}{(n+1)(n+2)},\\
&= \frac{3}{(n+1)(n+2)},\\
&\geq 0.
\end{align*}
So \((b_n)\) is increasing. Hence, by the above theorem, \((b_n)\) converges, and by the \emph{Algebra of Limits}, \[\sup(B) = \lim_{n \to \infty} b_n = \lim_{n \to \infty} \left(2 - \frac{\frac{3}{n}}{1 + \frac{1}{n}}\right) = 2,\]
as expected!
\EndKnitrBlock{solution*}

\end{document}
